\documentclass{aiaa-tc}

\usepackage{color}
\usepackage{amsmath}
%\usepackage{overcite}
\usepackage{graphicx}
\usepackage{subfig}
\usepackage{authblk}
\usepackage{amsfonts}

\input basic.ltx
\def\directory{EPSF/}

%------------------------------------------------------------------------------
% MAS Additions
\newcommand{\mas}[1]{\textcolor{magenta}{#1}}
\newcommand{\qw}[1]{\textcolor{blue}{#1}}
%useful for showing deleted text
\renewcommand{\kill}[1]{\textcolor{red}{\sout{#1}}}                             

\usepackage[normalem]{ulem}  % mas addition
\newcommand{\uvec}[1]{\bar{#1}}
\newcommand{\tens}[1]{\underline{\underline{#1}}}
\renewcommand{\vec}[1]{\underline{#1}}
\renewcommand{\skew}[1]{\widetilde{#1}}
%------------------------------------------------------------------------------

\title{Partitioned Nonlinear Structural Analysis\\
of Wind Turbines using BeamDyn}

\author[]{Qi Wang\thanks{Research Engineer, National Wind Technology Center, AIAA Member. Email: Qi.Wang2@nrel.gov}}
 \author[]{Michael A. Sprague\thanks{Senior Research Scientist, 
Computational Science Center.}}
 \author[]{Jason Jonkman\thanks{Senior Engineer, National Wind Technology Center, AIAA Professional Member.}}
  \author[]{Bonnie Jonkman\thanks{Senior Engineer, National Wind Technology Center.}}
 \affil[]{National Renewable Energy Laboratory, Golden, CO 80401}
 
 \renewcommand\Authands{, and }

\begin{document}

\maketitle

\begin{abstract}
{We present the numerical implementation of BeamDyn, a finite element beam solver based on geometrically exact beam theory, and its coupling to different modules in the FAST modularization framework. 
After reviewing the underlying BeamDyn theory and implementation, we describe the coupling algorithm and numerical integration scheme specifically designed for wind turbine analysis. 
The loose coupling method between BeamDyn and other modules is used where the nonlinear input-output coupling relations are solved by the Newton-Raphson method. 
Finally, we provide numerical examples to verify both BeamDyn and the coupling algorithm. }     
\end{abstract}

\section{Introduction} 
Recently, researchers at the National Renewable Energy Laboratory (NREL) 
developed BeamDyn, a nonlinear structural dynamics module for composite wind turbine blade analysis, in the FAST modularization framework. 
%This work is motivated by a fact that composite materials have been introduced into the wind energy industry which makes manufacturing of much lighter, larger wind turbine blade a possibility. 
%However, the use of composites also introduces some difficulties in the design and analysis stage. 
%For example, the coupling behavior that exists in composite materials can hardly be captured by the conventional theories which are usually based on isotropic assumptions. 
%Moreover, as the increasing size of the wind turbines, the deformations of the blades cannot be assumed to be linear anymore. 
BeamDyn, which is founded on geometrically exact beam theory (GEBT), was created to acurately simulate the large, nonlinear deflections of modern wind turbine blades that are designed with aero-elastic tailoring and complicated composite structures.
%which is a beam-deformation model useful in efficient analysis of highly flexible composite structures. 
GEBT was first proposed by Reissner\cite{Ressiner1973} and then extended to three-dimensional (3D) beams by Simo\cite{Simo1985} and Simo and Vu-Quoc\cite{Simo1986}. 
Readers are referred to
Hodges,\cite{HodgesBeamBook} in which comprehensive derivations and discussions
on nonlinear composite-beam theories can be found. 
In the BeamDyn implementation, the GEBT equations are discrretized spatially with Legendre spectral finite elements (LSFEs), which are $p$-type high-order elements that combine the accuracy of global spectral methods with the geometric modeling flexibility of low-order $h$-type elements.
More details on BeamDyn can be found in Wang et al.\cite{Wang:GEBT2014}

FAST is NREL's flagship multi-physics engineering tool for analyzing both land-based and offshore wind turbines under realistic operating conditions.  
The established FAST blade model included in the module Elastodyn is incapable 
of predictive analysis of highly flexible, composite wind turbine blades. 
FAST has been reformulated under a new modularized framework that provides a rigorous means by which various mathematical systems are implemented in distinct modules and coupled to other modules. 
The restructuring of FAST greatly enhanced flexibility and expandability to enable further developments of functionality without the need to recode established modules. 
These modules are interconnected to solve for the globally coupled dynamic responses of wind turbines and wind plants. \cite{Jonkman:2013,website:FASTModularizationFramework} In previous work, this framework is extended to handle the nonmatching spatial grids at interfaces and nonmatching temporal meshes that allow module solutions to advance with different time increments and different time integrators.\cite{Sprague:FAST2015}

In this paper, the formulation of the beam theory is first reviewed and then the coupling algorithm of BeamDyn with other modules is presented. 
%The numerical integration scheme that specifically implemented for wind turbine analysis is also introduced. 
Numerical examples are provided to verify the proposed beam solver and its coupling algorithm when it is running in a coupled-to-FAST mode.

\section{Formulation and Implementation}

\subsection{Geometrically Exact Beam Theory}





This section briefly reviews the geometrically exact beam theory. Further details on the content of this section can be found in many other papers\cite{YuGEBT} and textbooks. \cite{HodgesBeamBook,Bauchau:2010}
Figure~\ref{Kinematics} shows a beam in its initial undeformed
and deformed states. A reference frame $\mathbf{b}_i$ is introduced along the
beam axis for the undeformed state and a frame $\mathbf{B}_i$ is introduced
along each point of the deformed beam axis. The curvilinear coordinate $x_1$ defines the intrinsic parameterization of the reference line.
\begin{figure}
\centering
\includegraphics[clip,width=5.0in]{\directory Kinematics.pdf}
\caption{A beam deformation schematic.} \label{Kinematics}
\end{figure}
In this paper, matrix notation is used to denote vectorial or vectorial-like quantities. For example, an underline denotes a vector $\underline{u}$, a bar denotes a unit vector $\bar{n}$, and a double underline denotes a tensor $\underline{\underline{\Delta}}$. Note that sometimes the underlines only denote the dimension of the corresponding matrix. The governing equations of motion for geometrically exact beam theory (without structural damping) can be written as \cite{Bauchau:2010}
\begin{align}
	\label{GovernGEBT-1}
	\dot{\underline{h}} - \underline{F}^\prime &= \underline{f} \\
	\label{GovernGEBT-2}
	\dot{\underline{g}} + \dot{\tilde{u}} \underline{h} - \underline{M}^\prime + (\tilde{x}_0^\prime + \tilde{u}^\prime)^T \underline{F} &= \underline{m}
\end{align}
where $\vec{h}$ and $\vec{g}$ are the linear and angular momenta resolved in the inertial coordinate system, respectively; $\vec{F}$ and $\vec{M}$ are the beam's sectional force and moment resultants, respectively; $\vec{u}$ is the one-dimensional (1D) displacement of a point on the reference line; $\vec{x}_0$ is the position vector of a point along the beam's reference line;  and $\vec{f}$ and $\vec{m}$ are the distributed force and moment applied to the beam structure.  The notation $(\bullet)^\prime$ indicates a derivative with respect to beam axis $x_1$ and $\dot{(\bullet)}$ indicates a derivative with respect to time. The tilde operator $(\skew{\bullet})$ defines a skew-symmetric tensor corresponding to the given vector. In the literature, it is also termed as ``cross-product matrix." For example,
\[
	\skew{n} = 
	     		\begin{bmatrix}
			0 & -n_3 & n_2 \\
			n_3 & 0 & -n_1 \\
			-n_2 & n_1 & 0\\
			\end{bmatrix}	
\]
The constitutive equations relate the velocities to the momenta and the 1D strain measures to the sectional resultants as
\begin{align}
	\label{ConstitutiveMass}
	\begin{Bmatrix}
	\underline{h} \\
	\underline{g}
	\end{Bmatrix}
	= \underline{\underline{\mathcal{M}}} \begin{Bmatrix}
	\dot{\underline{u}} \\
	\underline{\omega}
	\end{Bmatrix} \\
	\label{ConstitutiveStiff}
	\begin{Bmatrix}
	\underline{F} \\
	\underline{M}
	\end{Bmatrix}
	= \underline{\underline{\mathcal{C}}} \begin{Bmatrix}
	\underline{\epsilon} \\
	\underline{\kappa}
	\end{Bmatrix}
\end{align}
where $\underline{\underline{\mathcal{M}}}$ and
$\underline{\underline{\mathcal{C}}}$ are the $6 \times 6$ sectional mass
and stiffness matrices, respectively (note that they are not really tensors);
$\underline{\epsilon}$ and $\underline{\kappa}$ are the 1D strains and
curvatures, respectively; and, $\underline{\omega}$ is the angular velocity
vector that is defined by the rotation tensor $\underline{\underline{R}}$ as
$\underline{\omega} =
\mathrm{axial}(\dot{\underline{\underline{R}}}~\underline{\underline{R}}^T)$. The axial vector $\vec{a}$ associated with a second-order tensor $\tens{A}$ is denoted $\vec{a}=\mathrm{axial}(\tens{A})$ and its components are defined as
\begin{equation}
    \label{axial}
    \vec{a} = \mathrm{axial}(\tens{A})=\begin{Bmatrix}
    a_1 \\
    a_2 \\
    a_3
    \end{Bmatrix}
    =\frac{1}{2}
    \begin{Bmatrix}
    A_{32}-A_{23} \\
    A_{13}-A_{31} \\
    A_{21}-A_{12}
    \end{Bmatrix}
\end{equation}
The 1D strain measures are defined as
\begin{equation}
    \label{1DStrain}
    \begin{Bmatrix}
        \vec{\epsilon} \\
        \vec{\kappa}
    \end{Bmatrix}
    =
    \begin{Bmatrix}
        \vec{x}^\prime_0 + \vec{u}^\prime - (\tens{R} ~\tens{R}_0) \bar{\imath}_1 \\
        \vec{k}
    \end{Bmatrix}
\end{equation}
where $\vec{k} = \mathrm{axial}[(\tens{R R_0})^\prime (\tens{R R_0})^T]$ is the sectional
curvature vector resolved in the inertial basis and $\bar{\imath}_1$ is the unit
vector along $x_1$ direction in the inertial basis. Note that these
three sets of equations, including equations of motion
Eq.~\eqref{GovernGEBT-1} and \eqref{GovernGEBT-2}, constitutive equations
Eq.~\eqref{ConstitutiveMass} and \eqref{ConstitutiveStiff}, and kinematical
equations Eq.~\eqref{1DStrain}, provide a full mathematical description of beam elasticity problems. 

For a displacement-based finite-element implementation, there are six
degrees of freedom at each node: three displacement components and three
rotation components. Here, $\vec{q}$ denotes the elemental
displacement array as $\underline{q}^T=\left[
\underline{u}^T~~\underline{p}^T\right]$ where $\vec{u}$ is the
displacement and $\vec{p}$ is the rotation-parameter vector. The
acceleration array can thus be defined as $\underline{a}^T=\left[
\ddot{\underline{u}}^T~~ \dot{\underline{\omega}}^T \right]$. For nonlinear
finite-element analysis, the discretized forms of
displacement, velocity, and acceleration are written as
\begin{align}
	\label{DiscretizedDisp}
	\underline{q} (x_1) &= \underline{\underline{N}} ~\hat{\underline{q}}~~~~~~~~\underline{q}^T = \left[ \underline{u}^T~~\underline{p}^T \right] \\
	\label{DiscretizedVel}
	\underline{v}(x_1) &= \underline{\underline{N}}~\hat{\underline{v}}~~~~~~~~\underline{v}^T = \left[\underline{\dot{u}}^T~~\underline{\omega}^T \right] \\
	\label{DiscretizedAcc}
	\underline{a}(x_1) &= \underline{\underline{N}}~ \hat{\underline{a}}~~~~~~~~\underline{a}^T = \left[ \ddot{\underline{u}}^T~~\dot{\underline{\omega}}^T \right]	
\end{align}
where $\tens{N}$ is the shape function matrix and $(\hat{\cdot})$ denotes a
column matrix of nodal values.

\subsection{Spatial and Temporal Discretization}

As described above, the GEBT model is discretized in space with LSFEs\cite{Ronquist-Patera:1987}, which are a high-order $p$-type finite element (FE).   
Elements have $p+1$ nodes located at the Gauss-Lobatto-Legendre points, where $p$ is the polynomial order of the Lagrangian-interpolant basis functions.  
For a given number of nodes, we have shown\cite{Wang:SFE2013,Wang:GEBT2014} that LSFEs for GEBT-based models can be dramatically more accurate than low-order elements.   
As a tool in BeamDyn, an LSFE evaluated with an appropriate quadrature scheme provides the option to model a modern, highly flexible turbine blade with a single element.  
The choice of numerical quadrature is described in the next section.
The GEBT-LSFE equations are time-integrated with a second-order-accurate generalized-alpha algorithm that is equipped with user-defined numerical damping (see Wang et al.\cite{Wang:GEBT2014} for details).  

\subsection{Finite-Element Quadrature}

Numerical integration (quadrature) of the finite-element inner products over an element domain is required in the FE formulation.   
Typically, the quadrature rule employed in a finite-element implementation is Gauss-Legendre (GL), for which the number of quadrature points is chosen based on the polynomial order of the underlying FE basis functions.  
In the case where material properties vary significantly over an element domain, the accuracy of the quadrature is degraded, which can affect the overall accuracy of the solution.   If the number of quadrature points is fixed to the FE basis-function order, accuracy is increased by either increasing number of elements ($h$-refinement) or the order of the elements ($p$-refinement).  However, if the quadrature order is chosen for accurate evaluation of FE inner products, then the choice in FE resolution can be chosen based on overall solution accuracy.

The material sectional properties are defined discretely at $n_s$ stations along the beam axis.
BeamDyn is equipped with two quadrature options: Gauss-Legendre quadrature and trapezoidal-rule (TR) quadrature.  
For GL quadrature, $n_q= p + 1$, where $n_q$ is the number of quadrature points and $p$ is the order of the LSFE  (however, the GL quadrature point locations differ from the p+1 GLL nodal locations).  Material properties are linearly interpolated from the nearest-neighbor discrete stations. 
Depending on the nature of the material properties, an increase in the element order $p$ could instigate a dramatically different response, because the quadrature points may capture different material properties.
For TR quadrature, $n_q = n_s + (n_s-1) \times (j-1)
= ( n_s - 1 ) \times j + 1$ is user specified, where $j$ is a positive integer. 
Trapezoidal-rule quadrature enables a user to model a modern turbine blade defined by many cross-sectional property stations with few node points (i.e., $p \ll n_s$) while capturing all of the provided material properties.
For example, 
the widely used NREL 5-MW reference wind turbine blade is defined by 49 stations along the blade axis.   If one were using first-order FEs with a fixed quadrature scheme, at least 48 elements would be required to accurately capture the material data in the FE inner products.
BeamDyn, with the GEBT model and LSFE $p$-type discretization, is equipped to model a wind turbine blade with a \textit{single} element.  
LSFE discretization with TR quadrature 
is an effective modeling approach when the beam deformation can be described accurately with relatively few FE nodes, despite the large number of material-property stations.    However, for a given element order and $n_q \gg p$, solutions will be significantly more expensive than if $n_q \approx p$ because inner products are evaluated at least once per time step. 


%The numerical analysis of wind turbine blade features a large number of cross-sectional data along the blade, usually ranging from dozens to more than one hundred stations. A typical practice in dealing with those cross-sectional data is interpolating a subset of the data to the quadrature point, for example, Gauss point. However, there are two drawbacks of this method. One is that it is very difficult to include all the data given the large number of cross sections used in wind turbine modeling. 
%The widely used NREL 5-MW reference wind turbine blade, for example, provides 49 stations along the blade axis. If linearly discretizing this blade, 48 first-order finite elements, with two end points align with two adjacent stations, are needed to make sure that all the cross-sectional data being used in the simulation. The other drawback is that there is no rigorous way to interpolate the station data to the quadrature points. 
 
% Given the discussion above, we implemented a new trapezoidal quadrature into BeamDyn, see Eq.~\ref{eqn:Trapezoidal}.
% \begin{equation}
%     \label{eqn:Trapezoidal}
%     \int_a^b f(x) dx \approx \displaystyle \frac{1}{2} \sum_{k=1}^{N}(x_{k+1}-%x_k)\left[f(x_{k+1}) + f(x_k)\right] 
% \end{equation}  
% where $N$ is the total number of intervals. Generally, this is a over-integration scheme by assuming that each station point is a quadrature point regardless of the number of the finite element nodes. Based on this assumption, all the cross-sectional data are included in the calculation and no interpolation is needed since the quadrature points are exactly located on the cross-sectional station points. 

\subsection{Module-Coupling Algorithm}
%For clarification, ``implicit" here refers to a module that needs information from other modules before the solution can be time-advanced; the modules are thus tied together in a linear or nonlinear system that must be solved for time advancement. 

The FAST modularization framework\cite{Jonkman:2013,Sprague:2014,Sprague:FAST2015} was created to loosely couple multi-physics modules for time domain simulation.  
Each module's dependent variables are described as states that are either continuous-in-time, discrete-in-time, or are contraints.  
Modules interact through input-output relationships, where input and output quantities are derived from states, and where inputs and outputs are defined on spatial meshes, and a predictor-corrector algorithm is employed to improve stability and accuracy of the time-update of the coupled system.    
The modularization environment provides utilities for coupling nonmatching meshes in space and time.
A detailed description of the coupling algorithm employed here can be found in Sprague et al.\cite{Sprague:FAST2015}, which we summarize as follows for the simplified case where each module is advanced with the same time increment $\Delta t$, and where we ignore details regarding the mapping of information between modules with non-matching meshes.   
Assume that we know all states, inputs, and outpus at time $t$.  In order to advance the states of all modules from time $t$ to $t+ \Delta t$, 
\begin{enumerate}

\item Using linear or quadratic exptrapolation of known inputs, approximate the inputs at $t+\Delta t$.  

\item Update the states of all modules to $t + \Delta t$.  

\item Solve the global system of input-output equations at $t + \Delta t$.  Depending on the relationship between modules and the module output equations, this system can range from a simple transfer of information to a nonlinear-system solve.  

\item Either accept the states, intputs, and outputs, or apply a correction by repeating step (2) with the inputs calculated in Step (3), and then repeating Step (3).

\end{enumerate}
It was shown, using simple numerical examples, in Sprague et al.\cite{Sprague:2014,Sprague:FAST2015} that employing one or more correction steps can increase the accuracy of a the coupled simulation and can increase stability by permiting the use of larger time increments.  However, these increases in accuracy and stability must be weighed against the additional cost of additional ``update state'' calculations for each module.


\mas{CLEAN THE FOLLOWING UP}
Here, we describe the inputs and outputs of the BeamDyn module.
BeamDyn was designed for wind turbine blade modeling, and,  as such, its inputs are root motion (e.g., from the ElastoDyn structural-dynamics module accounting for the drivetrain, nacelle, and support structure, including displacements/rotations, linear and angular velocities, and linear and angular accelerations), point loads (forces and moments), and distributed loads (e.g., those from the AeroDyn aerodynamics module). 
BeamDyn root output are root reaction forces and moments and beam motions. 
The BeamDyn mesh defining distributed-load inputs is a contiguous number of two-node line elements with nodes located at the BeamDyn quadrature points and the beam end points.  
The mesh defining point-load inputs is composed of points collocated with the BeamDyn FE nodes.  
The root-motion input is a point mesh at the beam root. 
The beam motion output mesh is a contiguous number of two-node line elements with nodes located at the BeamDyn FE nodes and the beam end points. 

%\mas{The following is incomplete, but I think we can delete}
%
%The BeamDyn inputs and outputs at the root (where the structural coupling of BeamDyn occurs) can be written as
%\begin{align}
    %\label{BDInput}
    %\mathbf{u}_{BD}^T &= \left[ \vec{q}^T(0)~~~\vec{v}^T(0)~~~\vec{a}^T\right] \\
    %\label{BDOutput}
    %\mathbf{y}_{BD}^T &= \left[ \vec{f}^T~~~\vec{m}^T \right] = \mathbf{Y}_{BD}(\mathbf{x}_{BD},\mathbf{u}_{BD},t)
%\end{align}
%The inputs and outputs of the module that is being coupled to BeamDyn often have the form
%\begin{align}
    %\label{OtherInput}
    %\mathbf{u}_{O}^T &= \left[ \vec{f}^T~~~\vec{m}^T \right] \\
    %\label{OtherOutput}
    %\mathbf{y}_{O}^T &= \left[ \vec{q}^T~~~\vec{v}^T~~~\vec{a}^T\right] = \mathbf{Y}_{O}(\mathbf{x}_{O},\mathbf{u}_{O},t) 
%\end{align}
%where $\mathbf{x}_{BD}$ and $\mathbf{x}_{O}$ are the states for the BeamDyn module and the coupled module, respectively. They include the displacement and velocity quantities in a standard state-space formulation.
%The input-output equations can be written as
%\begin{align}
    %\label{IOEq1}
    %\mathbf{U}_1: ~~&\mathbf{u}_{BD} - \mathbf{Y}_{O}(\mathbf{x}_{O},\mathbf{u}_{O},t) = 0 \\
    %\label{IOEq2}
     %\mathbf{U}_2: ~~&\mathbf{u}_{O} - \mathbf{Y}_{BD}(\mathbf{x}_{BD},\mathbf{u}_{BD},t) = 0
%\end{align}
%Newton-Raphson method is adopted here to solve this nonlinear system: 
%\begin{equation}
    %\label{NREq}
    %\begin{bmatrix}
    %\frac{\partial \mathbf{U}_1}{\partial \mathbf{u}_{BD}}  &  \frac{\partial \mathbf{U}_{1}}{\partial \mathbf{u}_{O}} \\
    %\frac{\partial \mathbf{U}_2}{\partial \mathbf{u}_{BD}}  &  \frac{\partial \mathbf{U}_{2}}{\partial \mathbf{u}_{O}} 
    %\end{bmatrix}
    %\begin{Bmatrix}
     %\Delta \mathbf{u}_{BD} \\
     %\Delta \mathbf{u}_{O}
    %\end{Bmatrix} 
    %=
    %-
    %\begin{Bmatrix}
     %\mathbf{U}_1 \\
     %\mathbf{U}_2
    %\end{Bmatrix}
%\end{equation}
 %The Jacobian matrix $\mathbf{J}$ on the left-hand-side is computed numerically using the forward-difference formulae
 %\begin{equation}
     %\label{Jacobian}
     %\mathbf{J}_{ij} = \frac{1}{\epsilon}_j \left[\mathbf{U}_i(\mathbf{u}+\epsilon_j \mathbf{e}_j) - \mathbf{U}_i(\mathbf{u}) \right]
 %\end{equation}
 %where $\mathbf{e}_j$ is the unit vector in the $\mathbf{u}_j$ direction and $\epsilon_j$ represents a small increment.
 %
%%It is pointed out that the initial guessed values of the unknown inputs in the above nonlinear system are obtained by numerically interpolation (or extrapolation) based on the known quantities, i.e., the previous time step values. A corrector scheme has also introduced in the FAST modularization framework that can be used to improve the accuracy of initial guess. Moreover, some simplifications have been introduced to FAST glue code to improve the efficiency. For example, the state variables, $\vec{q}$ and $\vec{v}$,  in the inputs Eq.~\eqref{BDInput} and \eqref{OtherInput}, are not involved in the input-output solve in Eq.~\eqref{NREq} because states can only be changed in time marching (UpdateStates). Interested readers please refer to Sprague et al.\cite{Sprague:FAST2015} for more details on the predictor-corrector scheme for coupling and Jonkman\cite{Jonkman:2013} for the FAST modularization framework.
 
 
\section{Numerical Examples}
 
\subsection{Example 1: Partitioned Analysis}

This example verifies our numerical implementation and examines the accuracy and numerical behavior of the coupling algorithm. 
A BeamDyn beam is ``clamped'' to the mass in a spring-damper-mass (SDM) system as shown in Figure~\ref{fig:CoupledSystem}.     
Here, the relevant output from BeamDyn is translation reaction force, whereas its inputs are root translational displacement, velocity, and acceleration.  
The inputs and outputs are the same, but swapped, for the SDM system.
The time integrator in BeamDyn is a second-order implicit generalized-$\alpha$ algorithm (see Wang et al.\cite{Wang:2014}), and the time integrator for the SDM system is fourth-order Adams-Bashforth-Moulton (ABM4), which is a predictor-corrector algorithm.
The material properties, coordinate system, and geometric parameters can be found in Figure~\ref{fig:CoupledSystem}. 
The cross-section dimensions of the beam is $0.1$ m $\times$ 0.1 m. 
The natural frequency of the uncoupled mass-spring-damper system is $6.28$ rad/s and the first five natural frequencies for the uncoupled beam (in a clamped/cantilevered configuration) are $0.26, 1.72, 5.78, 22.62$, and $24.21$ rad/s, respectively, as determined by a refined ANSYS modal analysis. 
The first five natural frequencies of the coupled system, obtained by ANSYS modal analysis, are $0.25, 0.85, 1.80, 4.76$, and $9.34$ rad/s, respectively.

\begin{figure}
\centering
\includegraphics[clip,width=4.0in]{\directory CoupledSystem.pdf}
\caption{Schematic showing a 0.1 m $\times$ 0.1 m beam attached to a spring-mass-damper system with system properties and dimensions.} 
\label{fig:CoupledSystem}
\end{figure}
 
In the present study, the system has quiescent initial conditions with an initial displacement of $0.1$ m from the neutral position in the positive $Z$ direction (see Figure ~\ref{fig:CoupledSystem}). 
First we examined the stability of the proposed coupling algorithm for which the beam is discretized by a single fifth-order LSFE. 
Figure~\ref{fig:CoupledDTPC} shows the maximum stable time increments obtained by numerical experiments against the number of correction iterations in the predictor-coupling scheme.  
For this system, increasing the number of correction iterations increases the allowable time increment for stability.  However, the savings offered by a larger time increment must be considered against the additional computational cost of the correction steps.  
For example, adding one correction step makes each time step about twice as expensive.
Figure~\ref{fig:CoupledDTPC} also shows maximum time increments when numerical damping is included in the BeamDyn module, and we see that such damping can also increase stability of the coupled system.  We note, however, that the maximum stable time increments for the uncoupled BeamDyn and SDM models are XX and XX respectively.  Thus, the coupled-system algorithm is significantly more stiff.

 \begin{figure}
\centering
\includegraphics[clip,width=4.0in]{\directory dt_pc.pdf}
\caption{Maximum stable time increments as a function of the number of correction iterations in the predictor-corrector coupling algorithm for the beam-spring-mass-damper system. Results are shown with ($\rho_\infty = 0.9$) and without ($\rho_\infty = 1.0$) numerical damping in the BeamDyn module.
} 
\label{fig:CoupledDTPC}
\end{figure}
 
For verification, we analyzed this case in ANSYS using 20 BEAM188 elements (cantilevered to a point mass with spring and damper) and the time increment was $10^{-5}$ s. The Newmark-$\beta$ time integrator is adopted in ANSYS without numerical damping. 
With this well-refined mesh in both time and space, the ANSYS solution is used as the benchmark in the following comparisons. 
The root and tip displacements of the beam can be found in Figure~\ref{fig:E1Disp}. 
For the BeamDyn results, the beam was discretized by a single fifth-order element, and the time increment was  $2\times10^{-5}$ s. No numerical damping and correction step are introduced for these results.  
%to resolve all the frequencies. 
Good agreements can be observed between the benchmark solution and the BeamDyn results. 

\begin{figure}
    \centering
    \begin{tabular}{c}
    \subfloat[Root Displacement]{\label{fig:E1DispRoot}\includegraphics[clip,width=3.0 in]{\directory  Disp_Root.pdf}} \qquad
\subfloat[Tip Displacement]{\label{fig:E1DispTip}\includegraphics[clip,width=3.0in]{\directory  Disp_Tip.pdf}}\\
\end{tabular}
\caption{Beam root and tip displacement histories calculated with ANSYS BEAM188 elements and BeamDyn.}
\label{fig:E1Disp}
\end{figure} 

The root velocity and accelerations are also compared between the current model and the ANSYS benchmark solution in Figure~\ref{fig:E1VelAcc}. 
Again, good agreements are found in those quantities. 
It is noted that high-frequency fluctuations can be observed in the acceleration plot of BeamDyn results. 
%We believe that the fluctuation is the constraint on the time step size and numerical damping damps some of these out thus in turn helps increasing the time step size as shown in Figure~\ref{fig:CoupledDTPC}. 
Introducing numerical damping into the coupling algorithm could be a solution to further reduce such high frequency behavior, which is a subject for future work.
%and thus allow larger time step size in the future. 

 \begin{figure}
    \centering
    \begin{tabular}{c}
    \subfloat[Root Velocity]{\label{fig:E1VelRoot}\includegraphics[clip,width=3.0 in]{\directory  Vel_Root.pdf}} \qquad
\subfloat[Root Acceleration]{\label{fig:E1AccRoot}\includegraphics[clip,width=3.0in]{\directory  Acc_Root.pdf}}\\
\end{tabular}
\caption{Beam root velocity and acceleration histories calculated with ANSYS BEAM188 elements and BeamDyn.}
\label{fig:E1VelAcc}
\end{figure} 

Next we studied the numerical performance of BeamDyn. The accuracy and efficiency of the present results of the coupled system were examined by the root-mean-square (RMS) errors, which aggregates the magnitudes of the errors in predictions for various times into a single measure of predictive performance. The error was calculated using
\begin{equation}
\varepsilon_{RMS}=\sqrt{\frac{\sum_{k=0}^{n_{max}}[U_z^k-U_b(t^k)]^2}{\sum_{k=0}^{n_{max}}[U_b(t^k)]^2}}
\label{RMSdefi}
\end{equation} 
where $U_b(t)$ is the benchmark solution given by ANSYS. Figure~\ref{fig:ConvDTNode} shows the RMS error of the tip displacements along $Z$ direction as a function of total number of nodes for three different time increment sizes and number of predictor-correction times. The time increments for Case 1, Case 2, and Case 3 are $2E-5$, $5E-4$, and $1E-3$, respectively; and the number of corrections for the three cases are 0 (no correction), 2, and 3, respectively.
\begin{figure}
\centering
\includegraphics[clip,width=5.0in]{\directory RMS_dt_node.pdf}
\caption{Normalized RMS error of $U_z$ histories as a function of total number of nodes for three different time increments and number of corrections. The dashed line shows ideal second-order convergence.} 
\label{fig:ConvDTNode}
\end{figure}
It can be concluded that although other factors have been introduced into the simulation, exponential convergence rate, which is a feature of LSFEs, in space discritization can be observed. The fifth-order element (six nodes) being used in this case is a reasonable choice for a converged analysis. Moreover, results obtained by corrections are more accurate than those with smaller time increments but no correction.

\subsection{Example 2: NREL 5-MW Wind Turbine}
The second example is a coupled analysis of NREL 5-MW reference wind turbine system. The three blades are modeled by BeamDyn, and the scheme discussed in this paper  was implemented to couple BeamDyn to FAST. 

First, we examined the numerical performance of two different quadrature methods, Gauss and trapezoidal quadratures, on this realistic blade analysis. A static cantilevered blade under a uniformly distributed force of magnitude $1E+04$ N/m along the flap direction is analyzed. Figure~\ref{fig:5MWStaticTip} shows the results. Monotonous convergence can be observed for the trapezoidal quadrature results with an increasing number of nodes; however, the convergence curve for the Gauss quadrature is very jagged, depending on which stations are used and how these data are interpolated to the Gauss points. This can be observed from the mass calculation, as shown in Figure~\ref{fig:5MWMass}: the constant mass by trapezoidal quadrature, which includes all the cross-sectional data regardless of the finite element discretization, and the mass calculated by Gauss quadrature is more random depending on the different datasets used and the interpolation methods. It is noted that a scaling factor has been applied to the calculation of blade mass so that it is consistent with ElastoDyn.\cite{report:NREL5MW} Hence, the trapezoidal quadrature will be applied to all subsequent simulations in this section.
\begin{figure}
\centering
\includegraphics[clip,width=5.0in]{\directory 5MW_Static_Tip.pdf}
\caption{Tip deflections of a cantilevered NREL 5-MW blade under uniformly distributed load by Gauss and trapezoidal quadratures.} 
\label{fig:5MWStaticTip}
\end{figure}

\begin{figure}
    \centering
\includegraphics[clip,width=5.0in]{\directory  5MW_Mass_Scaled.pdf}
\caption{Total blade mass of an NREL 5-MW reference blade calculated by Gauss and trapezoidal quadratures.}
\label{fig:5MWMass}
\end{figure} 

Next, we studied the time step sizes required for stability of BeamDyn in stand-alone and couple-to-FAST modes. Figure~\ref{fig:5MWdt_node} shows the maximum time step size versus the number of nodes and the error bars indicate the minimum search range. In the stand-alone mode, we used FAST as the driver but turning off all the coupling options so that the blade rotated at a fixed speed under gravity. For the coupled-to-FAST case, we conducted a coupled aero-servo-elastic wind turbine analysis under a mean wind speed of $12~m/s^2$ with turbulence, which is a certification test case \#26 in the FAST archive. There are no numerical damping and predictor-corrections introduced in these simulations.  

\begin{figure}
    \centering
\includegraphics[clip,width=5.0in]{\directory  5MW_dt_node.pdf}
\caption{Maximum stable time increments vs number of FE nodes.}
\label{fig:5MWdt_node}
\end{figure} 

The convergence of the time integrator is also evaluated using the RMS error defined in Eq.~\eqref{RMSdefi}. Again, the fixed-speed spinning case under gravity that uses FAST as a driver code is studied for the convergence, see Figure~\ref{fig:5MWdt_rms}. The blade is discretized by a single fifth-order element, and the benchmark solution is obtained  with a time step size of $0.001$ seconds. Second-order convergence rate can be observed in the plot.

\begin{figure}
    \centering
\includegraphics[clip,width=5.0in]{\directory  5MW_dt_rms.pdf}
\caption{RMS error of tip edge displacement vs. time step size.}
\label{fig:5MWdt_rms}
\end{figure} 

Finally, we studied the performance of BeamDyn in the coupled FAST analysis. Figures~\ref{fig:5MW_Coupled_dt} and \ref{fig:5MW_Coupled_Mesh} show the tip flap displacement histories under different time and space discretizations. Note that all the quantities studied here are resolved in the body-attached blade reference coordinate system following the IEC standard, where the $X$ direction is toward the suction side of the airfoil, the $Y$ direction is toward the trailing edge, and the $Z$ direction is toward the blade tip from root. It can be observed that for this case, reasonable results can be obtained by the $2E-3$ s time increment and a single fifth-order element.  

\begin{figure}
    \centering
\includegraphics[clip,width=\textwidth]{\directory  5MW_Coupled_dt.pdf}
\caption{Blade tip deflection histories along flap direction obtained using different time increments.}
\label{fig:5MW_Coupled_dt}
\end{figure} 

\begin{figure}
    \centering
\includegraphics[clip,width=\textwidth]{\directory  5MW_Coupled_Mesh.pdf}
\caption{Blade tip deflection histories along flap direction obtained using different elements.}
\label{fig:5MW_Coupled_Mesh}
\end{figure} 

\begin{figure}
    \centering
    \begin{tabular}{c}
    \subfloat[Flap Displacement]{\label{fig:5MWTipX}\includegraphics[clip,width=\textwidth]{\directory  B1TipTDxr_ScaledMass.pdf}} \\
\subfloat[Edge Displacement]{\label{fig:5MWTipY}\includegraphics[clip,width=\textwidth]{\directory  B1TipTDyr_ScaledMass.pdf}}\\
\subfloat[Axial Displacement]{\label{fig:5MWTipZ}\includegraphics[clip,width=\textwidth]{\directory  B1TipTDzr_ScaledMass.pdf}}\\
\end{tabular}
\caption{Comparisons of blade 1 tip displacements between ElastoDyn and BeamDyn results}
\label{fig:5MWTip}
\end{figure}  

We compared the results obtained by BeamDyn with those obtained by ElastoDyn for this case. The tip displacements of blade 1 are shown in Figure~\ref{fig:5MWTip}. Good agreements can be observed. It is noted that because of the Trapeze effect and elastic stretching considered in BeamDyn, the mean value of the axial displacement in Figure~\ref{fig:5MWTipZ} by BeamDyn is different from ElastoDyn. It is pointed out that the off-diagonal terms in the mass matrix in BeamDyn have been removed for comparison with ElastoDyn. Figure~\ref{fig:5MWRootF} shows the root reaction forces calculated by BeamDyn and ElastoDyn. Again, reasonable agreement can be found in these quantities. It is noted that there are spikes in the $M_{pitch}$ and we are still looking into it and plan to resolve these in a future release.

\begin{figure}[h!t]
    \centering
    \begin{tabular}{c}
    \subfloat[$F_{flap}$]{\label{fig:5MWRootFx}\includegraphics[clip,width=0.5\textwidth]{\directory  B1RootFxr_ScaledMass.pdf}} \qquad
\subfloat[$F_{edge}$]{\label{fig:5MWRootFy}\includegraphics[clip,width=0.5\textwidth]{\directory  B1RootFyr_ScaledMass.pdf}}\\
\subfloat[$F_{axial}$]{\label{fig:5MWRootFz}\includegraphics[clip,width=0.5\textwidth]{\directory  B1RootFzr_ScaledMass.pdf}} \qquad
\subfloat[$M_{edge}$]{\label{fig:5MWRootMx}\includegraphics[clip,width=0.5\textwidth]{\directory  B1RootMxr_ScaledMass.pdf}}\\
\subfloat[$M_{falp}$]{\label{fig:5MWRootMy}\includegraphics[clip,width=0.5\textwidth]{\directory  B1RootMyr_ScaledMass.pdf}} \qquad
\subfloat[$M_{pitch}$]{\label{fig:5MWRootMz}\includegraphics[clip,width=0.5\textwidth]{\directory  B1RootMzr_ScaledMass.pdf}}
\end{tabular}
\caption{Comparisons of root reaction loads between ElastoDyn and BeamDyn results}
\label{fig:5MWRootF}
\end{figure} 

\section{Conclusion}

In this paper, we examined the beam theory and the coupling algorithm for partitioned mechanical system analysis. The governing equations for BeamDyn were reviewed and the coupling schemes between BeamDyn and other mechanical modules were explained. For wind turbine blade analysis, which features a large number of cross-sectional stations along the blade axis, we implemented a trapezoidal numerical integration so that all the cross-sectional data, including inertial and stiffness matrices, can be used in the analysis. The users do not need to pick certain stations in a large set of data neither numerically interpolates them, which usually introduces errors. This new spatial integration method and the coupling algorithm have been verified in the numerical examples. The results obtained by trapezoidal quadrature were monotonously converged while those obtained by conventional Gauss integration were randomly converged depending on the stations chosen and interpolation method. In the coupled analysis, exponential convergence rate can be seen in the BeamDyn results, which is a prominent feature of the LSFEs. In the full turbine analysis, reasonable agreement between BeamDyn and ElastoDyn results is found. More verification and validation results can be found in Guntar et al.\cite{Sri:SciTech2016}.
  
\section*{Acknowledgments} 

This work was supported by the U.S. Department of Energy under Contract No.\
DE-AC36-08GO28308 with the National Renewable Energy Laboratory. Funding for the work was provided by the DOE Office of Energy Efficiency and Renewable Energy, Wind and Water Power Technologies Office.   

\bibliographystyle{aiaa}
\bibliography{references}

\end{document}
