%&LaTeX
\chapter{Input Files}
\label{sec:InputFiles}
Users specify the blade model parameters; including its geometry, properties, and FE and output control parameters; via a primary BeamDyn input file and a blade property input file. 
When used in stand-alone mode, an additional driver input file is required. 
This driver file specifies inputs normally provided to BeamDyn by FAST, including simulation range, root motions (initial conditions), and externally applied loads.

No lines should be added or removed from the input files, except in tables where the number of rows is specified.

\section{Units}
BeamDyn uses the SI system (kg, m, s, N). 
Angles are assumed to be in radians unless otherwise specified.

\section{BeamDyn Driver Input File}
\label{sec:DriverInputFile}
The driver input file is only needed for the stand-alone version of BeamDyn and contains inputs that are normally set by FAST, and that are necessary to control the simulation for uncoupled models. 

The driver input file begins with two lines of header information, which is for the user but is not used by the software. 
If BeamDyn run in the stand-alone mode, the results output file will be prefixed with the same name of this driver input file.

A sample BeamDyn driver input file is given in Appendix \ref{sec:AppDriver}

\subsection{Simulation Control Parameters}
\param{t\_initial}  and \textbf{\textit{t\_final} } specify the starting time of the simulation and ending time of the simulation, respectively. 
\textbf{\textit{dt} } specifies the time step size.

\subsection{Gravity Parameters}
\textbf{\textit{Gx} }, \textbf{\textit{Gy} }, and \textbf{\textit{Gz} } specify the components of gravity vector along $X$, $Y$, and $Z$ directions in the global coordinate system, respectively.  
In FAST, this is normally 0, 0, and -9.80665.

\subsection{Inertial Frame Parameters}
This section defines the relation between two inertial frames, the global coordinate system and initial blade reference coordinate system. 
\textbf{\textit{GlbPos(1)} }, \textbf{\textit{GlbPos(2)} }, \textbf{\textit{GlbPos(3)} } specifies three components of the initial global position vector along $X$, $Y$, and $Z$ directions resolved in the global coordinate system, see Figure \ref{fig:Frame}. 
And the following $3 \times 3$ direction cosine matrix (DCM) relates the rotations from global coordinate system to blade coordinate system. 

\begin{figure}[h!tp]
    \centering
    \includegraphics[width = 0.8\textwidth]{\directory Frame.jpg}
    \caption{Global and blade coordinate systems in BeamDyn}
    \label{fig:Frame}
\end{figure}

\subsection{Blade Floating Reference Frame Parameters}

This section specifies the parameters that defines the blade floating reference frame, which is a body-attached floating frame and the blade root is cantilevered at the origin of this frame. 
The floating blade reference fame is assumed to be in a constant rigid-body rotation mode about the origin of the global coordinate system, that is,
\begin{equation}
   \label{RootVelocity}
   v_r = \omega_r \times r_t
\end{equation}
where $v_r$ is the root (origin of the floating blade reference frame) translational velocity vector; $\omega_r$ is the constant root (origin of the floating blade reference frame) angular velocity vector; and $r_t$ is the global position vector introduced in the previous section at instant $t$, see Figure~\ref{fig:Frame}.  \mas{The figure implies more general motion, no???}
The floating blade reference frame coincides with the initial floating blade reference frame at the beginning $t=0$. 
\param{RootVel(4)}, \param{RootVel(5)}, and \param{RootVel(6)} specify the three components of the root angular velocity vector about  $X$, $Y$, and $Z$ axises in global coordinate system, respectively.  
\param{RootVel(1)}, \param{RootVel(2)}, and \param{RootVel(3)}, which are the three components of the root translational velocity vector along  $X$, $Y$, and $Z$ directions in global coordinate system, respectively, are calculated based on Eq.~\ref{RootVelocity}. 
BeamDyn can handle more complicate root motions by changing, for example, in $BD\_InputSolve$ subroutine in the Drvier\_Beam.f90:
\begin{lstlisting}[frame=single,language=Fortran]
   u%RootMotion%RotationVel(:,:) = 0.0D0
   u%RootMotion%RotationVel(1,1) = IniVelo(5)
   u%RootMotion%RotationVel(2,1) = IniVelo(6)
   u%RootMotion%RotationVel(3,1) = IniVelo(4)
   u%RootMotion%TranslationVel(:,:) = 0.0D0
   u%RootMotion%TranslationVel(:,1) = &
   MATMUL(BD_Tilde(real(u%RootMotion%RotationVel(:,1),BDKi)),temp_rr)
\end{lstlisting}
where $IniVelo(5)$, $IniVelo(6)$, and $IniVelo(4)$ are the three components of the root angular velocity vector about  $X$, $Y$, and $Z$ axising in the global coordinate system, respectively; $temp$\_$rr$ is the global position vector at instant $t$.

The blade is initialized in the rigid-body motion mode, i.e., based on the root velocity information defined in this section and the position information defined in the previous section, the motion of other points along the blade are initialized as
\begin{align}
    \label{IniRootAcc}
    a_{r0} &= \omega_r \times (\omega_r \times r_0) \\
    \label{IniTraVel}
    v_0 &= v_{r0} + \omega \times P \\
    \label{IniAngVel}
    \omega_0 &= \omega_r
\end{align}
where $a_{r0}$ is the initial root translational acceleration vector; $v_0$ and $\omega_0$ the initial translational and angular velocity vectors along blade other than the root, respectively; and $P$ is the position vector along the blade relative to the root. 

\subsection{Applied Load}
This section defines the applied loads, including distributed and tip-concentrated loads, for the analysis. 
The first six entries \param{DistrLoad(i)}, $i \in [1,6]$, specify three components of uniformly distributed force vector and three components of uniformly distributed moment vector in the global coordinate systems, respectively. 
The following six entries \param{TipLoad(i)}, $i \in [1,6]$, specify three components of concentrated tip force vector and three components of concentrated tip moment vector in the global coordinate system, respectively. 
The distributed load defined in this section is assumed to be uniform along the blade and constant throughout the simulation; the tip load is a constant concentrated load applied at the tip of a blade. 

BeamDyn is capable of handling more complex loading cases, e.g., time-dependent loads, through customization of the source code (requiring a recompile of stand-alone BeamDyn). 
The user can define such loads in the \textit{BD\_InputSolve} solve in the Driver\_Beam.f90 file, which is called every time step. 
The following section can be modified to define the concentrated load at each FE node:

\begin{lstlisting}[frame = single]
   ! Define concentrated force vector
   u%PointLoad%Force(:,:)  = 0.0D0
   ! Define concentrated moment vector
   u%PointLoad%Moment(:,:) = 0.0D0
\end{lstlisting}
where the first index in each array ranges from 1 to 3 for load vector components along three directions and the second index of each array ranges from 1 to  \param{node\_total}, where the latter is the total number of FE nodes. 
For example, a time-dependent sinusoidal force acting along the $X$ direction applied at the $2^{nd}$ FE node can be defined as
\begin{lstlisting}[frame = single]
   ! Define concentrated force vector
   u%PointLoad%Force(:,:) = 0.0D0
   u%PointLoad%Force(1,2)  = 1.0D+03*SIN((2.0*pi)*t/6.0 )
   ! Define concentrated moment vector
   u%PointLoad%Moment(:,:) = 0.0D0
\end{lstlisting}
with $1.0D+03$ being the amplitude and $6.0$ being the period.

Similar to the concentrated load, the distributed loads can be defined in the same subroutine
\begin{lstlisting}[frame = single]
   IF(p%quadrature .EQ. 1) THEN
       DO i=1,p%ngp*p%elem_total+2
           u%DistrLoad%Force(1:3,i) = InitInput%DistrLoad(1:3)
           u%DistrLoad%Moment(1:3,i)= InitInput%DistrLoad(4:6)
       ENDDO
   ELSEIF(p%quadrature .EQ. 2) THEN
       DO i=1,p%ngp
           u%DistrLoad%Force(1:3,i) = InitInput%DistrLoad(1:3)
           u%DistrLoad%Moment(1:3,i)= InitInput%DistrLoad(4:6)
       ENDDO
   ENDIF
\end{lstlisting}
where p\%ngp is the number of quadrature points, InitInput\%DistrLoad(:) is the constant uniformly distributed load BeamDyn reads from the driver input file, and p\%elem\_total is the total number of elements. 
The user can modify ``InitInput\%DistrLoad(:)'' to define the loads based on need. 

We note that the distributed loads are defined at the quadrature points for numerical integrations. 
For example, if Gauss quadrature is chosen (i.e., p\%quadrature .EQ. 1), then the distributed loads are defined at Gauss points plus the two end points of the beam (root and tip). 

\subsection{Primary Input File}
\param{InputFile} is the file name of the primary BeamDyn input file. 
This name should be in quotations and can contain an absolute path or a relative path. 

\section{BeamDyn Primary Input File}

The BeamDyn primary input file defines the blade geometry, LSFE-discretization and simulation options, output channels, and name of the blade input file. 
The geometry of the blade is defined by key-point coordinates and initial twist angles (in units of degree) in the blade local coordinate system (IEC standard blade system where Z$_r$ is along blade axis from root to tip, X$_r$ directs normally toward the suction side, and Y$_r$ directs normally toward the trailing edge).

The file is organized into several functional sections. 
Each section corresponds to an aspect of the BeamDyn model.

A sample BeamDyn primary input file is given in Appendix \ref{sec:AppPrimary}.

The primary input file begins with two lines of header information, which are for the user but are not used by the software.

\subsection{Simulation Controls}

The user can set the \param{Echo} flag to ``TRUE" to have BeamDyn echo the contents of the BeamDyn input file (useful for debugging errors in the input file). 

\param{Analysis\_Type} specifies the type of an analysis. 
In the current version, there are two options: 1) static analysis, and 2) dynamic analysis. 
If BeamDyn is run in coupled FAST mode, this entry can be only set to 2, i.e., for dynamic analysis.

\param{rhoinf} specifies the numerical damping parameter (spectral radius of the amplification matrix) in the range of $[0.0,1.0]$ used in the generalized-$\alpha$ time integrator implemented in BeamDyn for dynamic analysis. 
For $\textbf{\textit{rhoinf}}  = 1.0$, no numerical damping is introduced and the generalized-$\alpha$ scheme is identical to the Newmark scheme; for $\textbf{\textit{rhoinf}}  = 0.0$, maximum numerical damping is introduced. 
Numerical damping may help produce numerically stable solutions.

\textbf{\textit{Quadrature}} specifies the spatial numerical integration scheme. 
There are two options: 1) Gauss quadrature; and 2) Trapezoidal quadrature. 
We note that in the current version, Gauss quadrature is implemented in reduced form to improve efficiency and avoid shear locking. 
In the trapezoidal quadrature, only one member (FE element) can be defined in the following GEOMETRY section of the primary input file.  
Trapezoidal quadrature is appropriate when the number of ``blade input stations'' (described below) is significantly greater than the order of the LSFE.

\param{Refine} specifies a refinement parameter used in trapezoidal quadrature. 
An integer value greater than unity will split the space between two input stations into ``Refine factor'' of segments.
The keyword ``DEFAULT'' may be used to set it to 1, i.e., no refinement is needed. 
This entry is not used in Gauss quadrature.   

\param{N\_Fact} specifies a parameter used in the modified Newton-Raphson scheme. 
If $\param{N\_Fact} = 1$ a full Newton iteration scheme is used, i.e., the global tangent stiffness matrix is computed and factorized at each iteration; if $\param{N\_Fact} >  1$ a modified Newton iteration scheme is used, i.e., 
the global stiffness matrix is computed and factorized every \param{N\_Fact} iterations within each time step. 
The keyword ``DEFAULT'' set $\textbf{\textit{N\_Fact}} = 5$.

\param{DTBeam} specifies the constant time increment of the time-integration in seconds. 
The keyword ``DEFAULT'' may be used to indicate that the module should employ the time increment prescribed by the driver code (FAST/stand-alone driver program).

\param{NRMax} specifies the maximum number of iterations in the Newton-Raphson scheme. 
If convergence is not reached within this number of iterations, BeamDyn returns an error message and terminates the simulation. 
The keyword ``DEFAULT'' sets $\textbf{\textit{NRMax}} = 10$.

\param{Stop\_Tol} specifies a tolerance parameter used in convergence criteria of a nonlinear solution that is used for the termination of the iteration. 
The keyword ``DEFAULT'' sets $\textbf{\textit{Stop\_Tol}} = 1.0E-05$.  \mas{Please elaborate on the nature of the convergence criteria}

\subsection{Geometry Parameter}

The blade beam model is composed of several {\it members} in contiguous series and each member is defined by at least three key points in BeamDyn. 
A cubic-spline-fit pre-processor implemented in BeamDyn automatically generates the member based on the key points and then interconnects the members into a blade. 
There is always a shared key point at adjacent members; therefore the total number of key points is related to number of members and key points in each member.

\param{Member\_Total} specifies the total number of beam members used in the structure. 
With the LSFE discretization, a single member and a sufficiently high element order, \param{Order\_Elem}, may well be sufficient.

\param{KP\_Total} specifies the total number of key points used to define the beam members.   

The following section contains two header lines and \param{Member\_Total} lines. 
Each line has two integers providing the member number (must be 1, 2, 3, etc., sequentially) and the number of key points in this member, respectively. 
It is noted that the number of key points in each member is not independent of the total number of key points and they should satisfy the following equality:
\begin{equation}
    \label{Keypoint}
    KP_{t} = \sum_{i=1}^{M_t} n_i - M_t +1
\end{equation}
where $KP_{t}$ is the total number of key points, $M_t$ is the total number of members, and $n_i$ is the number of key points in the $i^{th}$ member. 
Because cubic splines are implemented in BeamDyn, $n_i$ must be greater than or equal to three. 
Figure \ref{fig:Geometry1} shows two cases for member and key-point definition.
\begin{figure}[h!tp]
\centering
    \subfigure[Case 1 \label{Geometry1:Case1}]{\includegraphics[width=4.0in]{\directory Geometry_Member1.png}}\\
    \subfigure[Case 2 \label{Geometry1:Case2}]{\includegraphics[width=4.0in]{\directory Geometry_Member2.png}}
    \caption{Member and key point definition. Case 1: one member defined by four key points; and Case 2: two members defined by six key points.}
    \label{fig:Geometry1}
\end{figure}

The next section defines the key-point information. 
Each key point is defined by three physical coordinates in the IEC standard blade coordinate system along with a structural twist angle in the unit of degrees. 
The structural twist angle is also following the IEC standard which is defined as the twist about the negative Z axis. 
The key points are entered sequentially and there should be a total of \param{KP\_Total} lines for BeamDyn to read in the information, after two header lines.

\subsection{Mesh Parameter}

\param{Order\_Elem} specifies the order of shape functions for each finite element.  
Each LSFE will have \param{Order\_Elem}+1 nodes located at the GLL quadrature points.
All LSFEs will have the same order.
With the LSFE discretization, an increase in accuracy will, in general, be better achieved by increasing \param{Order\_Elem} (i.e., $p$-refinement) rather than increasing the number of members (i.e., $h$-refinement).
For Gauss quadrature, \param{Order\_Elem} should be greater than one. 

\subsection{Material Parameter}

\param{BldFile} is the file name of the blade input file. 
This name should be in quotations and can contain an absolute path or a relative path.

\subsection{Pitch Actuator Parameter}

In this release, the pitch actuator implemented in BeamDyn is not available. 
The \textbf{\textit{UsePitchAct}} should be set to "FALSE" in this version, whereby the input blade-pitch angle prescribed by the driver code is used to orient the blade directly. 
\param{PitchJ}, \param{PitchK}, and \param{PitchC} specify the pitch actuator inertial, stiffness, and damping coefficient, respectively. 
In future releases, specifying \param{UsePitchAct} $=$ TRUE will enable a second-order pitch actuator, whereby the pitch angular orientation, velocity, and acceleration are determined by the actuator based on the input blade-pitch angle prescribed by the driver code.

\subsection{Outputs}

In this section of the primary input file, the user sets flags and switches for the desired output behavior.

Specifying \param{SumPrint} $=$ TRUE causes BeamDyn to generate a summary file with name \param{PrimaryInputFile}.\textit{sum}. 
See Section~\ref{sec:SumFile} for summary file details.

\param{OutFmt} parameter controls the formatting of the results within the stand-alone BeamDyn output file. 
It needs to be a valid Fortran format string, but BeamDyn currently does not check the validity. 
This input is unused when BeamDyn is used coupled to FAST.

\param{NNodeOuts} specifies the number of nodes to output to file. 
Currently, BeamDyn can output quantities at a maximum of nine nodes. 

\param{OutNd} is a list \param{NNodeOuts} long of node numbers between 1 and $node_total$ (total number of FE nodes), separated by any combination of commas, semicolons, spaces, and/or tabs.   
The nodal positions are given in the summary file, if output.

The \param{OutList} block contains a list of output parameters. 
Enter one or more lines containing quoted strings that in turn contain one or more output parameter names.  
Separate output parameter names by any combination of commas, semicolons, spaces, and/or tabs.  
If you prefix a parameter name with a minus sign, ``-", underscore, ``\_", or the characters ``m" or ``M", BeamDyn will multiply the value for that channel by �1 before writing the data.  
The parameters are written in the order they are listed in the input file.  
`BeamDyn allows you to use multiple lines so that you can break your list into meaningful groups and so the lines can be shorter. 
You may enter comments after the closing quote on any of the lines.  
Entering a line with the string ``END'' at the beginning of the line or at the beginning of a quoted string found at the beginning of the line will cause BeamDyn to quit scanning for more lines of channel names.  
Node-related quantities are generated for the requested nodes identified through the OutNd list above.  
If BeamDyn encounters an unknown/invalid channel name, it warns the users but will remove the suspect channel from the output file. 
Please refer to Appendix \ref{sec:AppOutputChannel} for a complete list of possible output parameters and their names.

\section{Blade Input File}

The blade input file defines the cross-sectional properties at various stations along a blade and six damping coefficient for the whole blade. 
A sample BeamDyn blade input file is given in Appendix \ref{sec:AppBlade}. 
The blade input file begins with two lines of header information, which is for the user but is not used  by the software.

\subsection{Blade Parameters}

\param{Station\_Total} specifies the number cross-sectional stations along the blade axis used in the analysis.

\param{Damp\_Type} specifies if structural damping is considered in the analysis. 
If \param{Damp\_Type} $= 0 $, then no damping is considered in the analysis and the six damping coefficient in the next section will be ignored. 
If \param{Damp\_Type} $ = 1$, structural damping will be included in the analysis.

\subsection{Damping Coefficient}

This section specifies six damping coefficients, $\mu_{ii}$ with $i \in [1,6]$, for six DOFs (three translations and three rotations). 
Viscous damping is implemented in BeamDyn where the damping forces are proportional to the strain rate. 
These are stiffness-proportional damping coefficients, whereby the $6\times6$ damping matrix at each cross section is scaled from the $6 \times 6$ stiffness matrix by these diagonal entries of a $6 \times 6$ scaling matrix:
\begin{equation}
    \label{DampingForce}
    \mathcal{\underline{F}}^{Damp} = \underline{\underline{\mu}}~\underline{\underline{S}}~\dot{\underline{\epsilon}} 
\end{equation}
where $\mathcal{\underline{F}}^{Damp}$ is the damping force, $\underline{\underline{S}}$ is the $ 6 \times 6$ cross-sectional stiffness matrix, $\dot{\underline{\epsilon}} $ is the strain rate, and $\underline{\underline{\mu}}$ is the damping coefficient matrix defined as
\begin{equation}
   \label{DampMatrix}
   \underline{\underline{\mu}} = 
   \begin{bmatrix}
       \mu_{11} & 0 & 0 & 0 & 0 & 0 \\
       0 & \mu_{22} & 0 & 0 & 0 & 0 \\
       0 & 0 & \mu_{33} & 0 & 0 & 0 \\
       0 & 0 & 0 & \mu_{44} & 0 & 0 \\
       0 & 0 & 0 & 0 & \mu_{55} & 0 \\
       0 & 0 & 0 & 0 & 0 & \mu_{66} \\
   \end{bmatrix}
\end{equation}

\subsection{Distributed Properties}

This section specifies the cross-sectional properties at each of the \textbf{\textit{Station\_Total}} stations. 
For each station, a non-dimensional parameter $\eta$ specifies the station location along the blade axis ranging from $[0.0,1.0]$. 
The first and last station parameters must be set to $0.0$ and $1.0$, respectively.

Following the station location parameter $\eta$, there are two $6 \times 6$ matrices providing the structural and inertial properties for this cross-section. 
First is the stiffness matrix and then the mass matrix. 
We note that these matrices are defined in a local coordinate system along the blade axis following the IEC blade coordinate convention with $Z_{rl}$ directing toward the unit tangent vector of the axis. 
For a cross-section without coupling effects, for example, the stiffness matrix is given as follows:
\begin{equation}
    \label{Stiffness}
    \begin{bmatrix}
    K_{ShrFlp} & 0 & 0 & 0 & 0 & 0 \\
    0 & K_{ShrEdg} & 0 & 0 & 0 & 0 \\
    0 & 0& EA & 0 & 0 & 0 \\
    0 & 0 & 0 & EI_{Edg} & 0 & 0 \\
    0 & 0 & 0 & 0 & EI_{Flp} & 0 \\
    0 & 0 & 0 & 0 & 0 & GJ
    \end{bmatrix}
\end{equation}
where $K_{ShrEdg}$ and $K_{ShrFlp}$ are the edge and flap shear stiffnesses, respectively; $EA$ is the extension stiffness; $EI_{Edg}$ and $EI_{Flp}$ are the edge and flap stiffnesses, respectively; and $GJ$ is the torsional stiffness.

A generalized sectional mass matrix is given by:
\begin{equation}
    \label{Mass}
    \begin{bmatrix}
    m & 0 & 0 & 0 & 0 & -m Y_{cm} \\
    0 & m & 0 & 0 & 0 & m X_{cm}\\
    0 & 0& m & m Y_{cm} & -m X_{cm} & 0 \\
    0 & 0 & m Y_{cm} & i_{Edg} & -i_{cp} & 0 \\
    0 & 0 &-m X_{cm} & -i_{cp} & i_{Flp} & 0 \\
    -m Y_{cm} & m X_{cm} & 0 & 0 & 0 & i_{plr}
    \end{bmatrix}
\end{equation}
where $m$ is the mass density per unit span; $X_{cm}$ and $Y_{cm}$ are the local coordinates of the sectional center of mass, respectively; $i_{Edg}$ and $i_{Flp}$ are the edge and flap mass moments of inertia per unit span, respectively; $i_{plr}$ is the polar moment of inertia per unit span; and $i_{cp}$ is the sectional cross-product of inertia per unit span. 
We note that for beam structure, the $i_{plr}$ is given as  (although this relationship is not checked by BeamDyn)
\begin{equation}
    \label{PolarMOI} %Polar Moment Of Inertia
    i_{plr} = i_{Edg} + i_{Flp}
\end{equation}


