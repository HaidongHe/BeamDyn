\appendix
\chapter{BeamDyn Driver Input File}
\label{sec:AppDriver}
\includepdf[pages = - ,pagecommand={\thispagestyle{plain}}]{\directory BDDriverInput.pdf}


\chapter{BeamDyn Primary Input File: NREL 5-MW Reference Wind Turbine}
\label{sec:AppPrimary}
\includepdf[pages = 1 - 3,pagecommand={\thispagestyle{plain}}]{\directory BDPrimaryInput.pdf}

\chapter{BeamDyn Blade Input File: NREL 5-MW Reference Wind Turbine Blade}
\label{sec:AppBlade}
\includepdf[pages = -, pagecommand={\thispagestyle{plain}}]{\directory BDBladeInput.pdf}

\chapter{BeamDyn List of Output Channels}
\label{sec:AppOutputChannel}
This is a list of all possible output parameters for the BeamDyn module.  The names are grouped by meaning, but can be ordered in the OUTPUTS section of the BeamDyn primary input file as the user sees fit.  N$\beta$, refers to output node $\beta$, where $\beta$ is a number in the range [1,9], corresponding to entry $\beta$ in the \textbf{\textit{OutNd}} list. When coupled to FAST, ``$B\alpha$" is prefixed to each output name, where $\alpha$ is a number in the range [1,3], corresponding to the blade number.  The outputs are expressed in one of the following three coordinate systems:
\begin{itemize}
    \item r: a floating reference coordinate system fixed to the root of the moving beam; when coupled to FAST for blades, this is equivalent to the IEC blade (b) coordinate system.
    \item l: a floating coordinate system local to the deflected beam.
    \item g: the global inertial frame coordinate system; when coupled to FAST, this is equivalent to FAST's global inertial frame (i) coordinate system.
\end{itemize}
\includepdf[pages = -, pagecommand={\thispagestyle{plain}}]{\directory BDOutputChannel.pdf}