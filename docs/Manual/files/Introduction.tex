\chapter{Introduction}
BeamDyn is a time-domain structural-dynamics module for slender structures created by the National Renewable Energy Laboratory (NREL) through U.S. Department of Energy Wind and Water Power Program support. The module has been coupled into the FAST aero-hydro-servo-elastic wind turbine multi-physics engineering tool where it used to model blade structural dynamics. BeamDyn is designed to analyze beams that are made of composite materials, initially curved and twisted, and subject to large displacement and rotation deformations. BeamDyn can also be used for static analysis of beams.

The new BeamDyn module follows the requirements of the FAST modularization framework, couples to FAST version 8, and provides new capabilities for modeling initially curved and twisted composite wind turbine blades undergoing large deformation. BeamDyn can also be driven as a stand-alone code to compute the static and dynamic responses of slender structures (blades or otherwise) under prescribed boundary conditions uncoupled from FAST.

BeamDyn is based on the geometrically exact beam theory (GEBT) and is implemented using Legendre spectral finite elements (LSFEs). GEBT supports full geometric nonlinearity and large deflection, with bending, torsion, shear, and extensional degree-of-freedom (DOFs); anisotropic composite material couplings (using full $6 \times 6$ mass and stiffness matrices, including bend-twist coupling); and a reference axis that permits blades that are not straight (supporting built-in curve, sweep, and sectional offsets). LFSEs are {\it p}-type elements that combine the accuracy of global spectral methods with the geometric modeling flexibility of the {\it h}-type finite elements (FEs). For smooth solutions, LSFEs have exponential convergence rates compared to low-order elements that have algebraic convergence. Two spatial numerical integration schemes, including the reduced Gauss quadrature and a trapezoidal over-integration, have been implemented for FE analysis. The trapezoidal scheme is specifically designed for wind turbine blade analysis, where the cross-sectional beam properties may vary dramatically such that a large number of cross-sectional stations along the blade axis is needed. In this scheme, information at all specified stations, including stiffness and mass constants, will be captured such that the common practice of interpolating those to the quadrature points, which is usually lack of sound theoretical foundations, can be avoided. Time integration of the BeamDyn equations of motion is achieved through the implicit generalized-$\alpha$ solver, with user-specified numerical damping. Given the theoretical foundation, powerful numerical tools introduced above, BeamDyn can solve the complicate nonlinear composite beam problem in an efficient manner, for example, finishing the coupled transient dynamic analysis of composite wind turbine blade by a single $5^{th}$-order LSFE with dozens of cross-sectional stations. 

When coupled with FAST, loads and responses are transferred between BeamDyn, ElastoDyn, ServoDyn, and AeroDyn via the FAST driver program (glue code) to enable aero-elasto-servo interaction at each coupling time step. There is a separate instance of BeamDyn for each blade. At the root node, the six DOFs displacements (three translations and three rotations), velocities, and accelerations are inputs to BeamDyn from ElastoDyn; and the six reaction loads (three force and three moment resultants) at the root of the wind turbine blade are outputs from BeamDyn to ElastoDyn, including the blade-pitch command through ServoDyn. BeamDyn also outputs the local blade displacements, velocities, and accelerations to AeroDyn in order to calculate the local aerodynamic applied loads that become inputs for BeamDyn. In addition, BeamDyn can calculate member internal reaction loads, as requested by the user. Please refers to Figure~\ref{fig:FlowChart} for the coupled interactions between BeamDyn and other modules in FAST. When uncoupled from FAST, the root motion and applied loads are specified via a stand-alone BeamDyn driver code.
\begin{figure}
    \centering
    \includegraphics[width = 6.0 in,angle = -90]{\directory FlowChart.eps}
    \caption{Coupled interaction between BeamDyn and FAST}
    \label{fig:FlowChart}
\end{figure}

The input file defines the blade geometry, cross-sectional material mass,stiffness, and damping properties, FE resolutions, and other simulation and output control parameters. The blade geometry is defined through a curvilinear blade reference axis by a series of key points in three-dimensional (3D) space along with the initial twist angles at these points. Each member contains at least three key points for the cubic spline fit implemented in BeamDyn; and is considered a LSFE with a parameter defining the order of the element. FE nodes, which are usually not evenly spaced along the element for LSFEs, will be generated by the module based on the mesh information. Blade properties are specified in a non-dimensional coordinate ranging from 0.0 to 1.0 along beam axis and are linear interpolated between two stations if needed  by the spatial integration method. 

The applied loads to BeamDyn can be either distributed loads at quadrature points or concentrated load at FE nodes or a combination of the the two types of loads.  When coupled to FAST, the aerodynamic loads computed by AeroDyn are transferred by the glue code at the nodes that are used for quadrature in BeamDyn.

This document is organized as follows. Section~\ref{sec:Run} details how to obtain the BeamDyn and FAST software archives and run either the stand-alone version of BeamDyn or BeamDyn coupled to FAST. Section~\ref{sec:InputFiles} describes the BeamDyn input files. Section~\ref{sec:OutputFiles} discusses the output files generated by BeamDyn. Example input files are shown in Appendix~\ref{sec:AppDriver}, \ref{sec:AppPrimary}, and  \ref{sec:AppBlade}. A summary of available output channels is found in Appendix~\ref{sec:AppOutputChannel}