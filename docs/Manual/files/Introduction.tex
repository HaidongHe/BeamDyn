\chapter{Introduction}
BeamDyn is a time-domain structural-dynamics module for slender structures created by the National Renewable Energy Laboratory (NREL) through support from the U.S.\ Department of Energy Wind and Water Power Program and the NREL Laboratory Directed Research and Development (LDRD) program through the grant ``High-Fidelity Computational Modeling of Wind-Turbine Structural Dynamics.''  \mas{Qi, I suggest you cite all our many papers on this topic}
The module has been coupled into the FAST aero-hydro-servo-elastic wind turbine multi-physics engineering tool where it used to model blade structural dynamics. 
%BeamDyn is designed to analyze beams that are made of composite materials, allowing for initial curvature and twist, and subject to large displacement and rotation deformations. BeamDyn can also be used for static analysis of beams.
The BeamDyn module follows the requirements of the FAST modularization framework (ADD REF), couples to FAST version 8, and provides new capabilities for modeling initially curved and twisted composite wind turbine blades undergoing large deformation. 
BeamDyn can also be driven as a stand-alone code to compute the static and dynamic responses of slender structures (blades or otherwise) under prescribed boundary conditions uncoupled from FAST.

The model underlying BeamDyn is the geometrically exact beam theory (GEBT) \mas{ADD REF}.   
GEBT supports full geometric nonlinearity and large deflection, with bending, torsion, shear, and extensional degree-of-freedom (DOFs); anisotropic composite material couplings (using full $6 \times 6$ mass and stiffness matrices, including bend-twist coupling); and a reference axis that permits blades that are not straight (supporting built-in curve, sweep, and sectional offsets). 
The GEBT beam equations are discretized in space with Legendre spectral finite elements (LSFEs).  
LFSEs are {\it p}-type elements that combine the accuracy of global spectral methods with the geometric modeling flexibility of the {\it h}-type finite elements (FEs). 
For smooth solutions, LSFEs have exponential convergence rates compared to low-order elements that have algebraic convergence. 
Two spatial numerical integration schemes are implemented for the finite element inner products: reduced Gauss quadrature and trapezoidal-rule integration.  
Trapezoidal-rule integration is appropriate when a large number of sectional properties are specified along the beam axis, for example, in a long wind turbine blade with material properties that vary dramatically over the length.  
Time integration of the BeamDyn equations of motion is achieved through the implicit generalized-$\alpha$ solver, with user-specified numerical damping.
The combined GEBT-LSFE  approach permits users to model a long, flexible, composite wind turbine blade with a single high-order element.  
Given the theoretical foundation, powerful numerical tools introduced above, BeamDyn can solve the complicate nonlinear composite beam problem in an efficient manner. For example, it was recently shown that a grid-independent dynamic solution of a 50 m composite wind turbine blade and with dozens of cross-section stations could be achieved with a 
single $5^{th}$-order LSFE \mas{Add Reference; SciTech Paper}

When coupled with FAST, loads and responses are transferred between BeamDyn, ElastoDyn, ServoDyn, and AeroDyn via the FAST driver program (glue code) to enable aero-elasto-servo interaction at each coupling time step. 
There is a separate instance of BeamDyn for each blade. 
At the root node, BeamDyn inputs are six displacements (three translations and three rotations), six velocities, and six accelerations; the root node outputs are six reaction forces (three translational forces and three moments). 
BeamDyn also outputs the blade displacements, velocities, and accelerations along the beam length, which are used by AeroDyn to calculate the local aerodynamic loads that are used as inputs for BeamDyn. 
In addition, BeamDyn can calculate member internal reaction loads, as requested by the user. 
Please refers to Figure~\ref{fig:FlowChart} for the coupled interactions between BeamDyn and other modules in FAST. 
When uncoupled from FAST, the root motion and applied loads are specified via a stand-alone BeamDyn driver code.
\begin{figure}
    \centering
    \includegraphics[width = \textwidth,angle = 0]{\directory FlowChart.jpg}
    \caption{Coupled interaction between BeamDyn and FAST}
    \label{fig:FlowChart}
\end{figure}

The BeamDyn input file defines the blade geometry, cross-sectional material mass, stiffness, and damping properties, FE resolution, and other simulation- and output-control parameters. 
The blade geometry is defined through a curvilinear blade reference axis by a series of key points in three-dimensional (3D) space along with the initial twist angles at these points. 
Each \textit{member} contains at least three key points for the cubic spline fit implemented in BeamDyn; each member is discretized with a single LSFE with a parameter defining the order of the element. 
Note that the number of key points defining the member and the order ($N$) of the LSFE are independent.
LSFE nodes, which are located at the $N+1$ Gauss-Legendre-Lobatto points, are not evenly spaced along the element; node locations are generated by the module based on the mesh information. 
Blade properties are specified in a non-dimensional coordinate ranging from 0.0 to 1.0 along beam axis and are linearly interpolated between two stations if needed by the spatial integration method. 
The BeamDyn applied loads can be either distributed loads specified at quadrature points,  concentrated loads specified at FE nodes, or a combination of the two.  

This document is organized as follows. Section~\ref{sec:Run} details how to obtain the BeamDyn and FAST software archives and run either the stand-alone version of BeamDyn or BeamDyn coupled to FAST. Section~\ref{sec:InputFiles} describes the BeamDyn input files. Section~\ref{sec:OutputFiles} discusses the output files generated by BeamDyn. Example input files are shown in Appendix~\ref{sec:AppDriver}, \ref{sec:AppPrimary}, and  \ref{sec:AppBlade}. A summary of available output channels is found in Appendix~\ref{sec:AppOutputChannel}
