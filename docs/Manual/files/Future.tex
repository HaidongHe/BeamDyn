\chapter{Future Work}
\label{sec:FutureWork}
The following list contains future work on BeamDyn software:

\begin{itemize}
   \item Eliminating numerical problems in single precision.
    \item Implementing eigenvalue analysis.
    \item Improving input options for stand-alone version to make it more user-friendly.
    \item Implementing GEBT based on modal method  for computational efficiency.
    \item Adding more options for blade cross-sectional properties inputs. For example, for general isotropic beams, engineering parameters including sectional offsets, material properties, etc will be used to generate the 6 $\times$ 6 matrices needed by BeamDyn.
    \item Writing a general guidance on modeling composite beam structures using BeamDyn, , for example, how to select a time step, how to select the model discretization, how to define the blade reference axis, where to get 6x6 mass/stiffness matrices, etc.
    \item Extending applications in FAST to other slender structures in the wind turbine system, for example, tower, mooring lines, and shaft.
    \item Developing a simplified form of GEBT with only rotational DOFs (bending, torsion) for computational efficiency.
\end{itemize}