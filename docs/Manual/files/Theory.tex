\chapter{BeamDyn Theory}
\label{sec:Theory}

This section focuses on the theory behind the BeamDyn module. 
The theoretical foundation, numerical tools, and some special handling in the implementation will be introduced. References will be provided in each section detailing the theories and numerical tools.
%\mas{If you think it appropriate, please add some good references where this material may also be found.}


In this chapter, matrix notation is used to denote vectorial or vectorial-like quantities. 
For example, an underline denotes a vector $\underline{u}$, an over bar denotes unit vector $\bar{n}$, and a double underline denotes a tensor $\underline{\underline{\Delta}}$. 
Note that sometimes the underlines only denote the dimension of the corresponding matrix.

\section{Coordinate Systems}
Figure~\ref{fig:BDFrame}  shows the coordinate system used in BeamDyn.
\begin{figure}[h!tp]
    \centering
    \includegraphics[width = 3.5 in]{\directory BDFrame.pdf}
    \caption{Global, blade reference, and internal coordinate systems in BeamDyn. Illustration by Al Hicks, NREL}
    \label{fig:BDFrame}
\end{figure}

\subsection{Global Coordinate System}
The global coordinate system is denoted as \textbf{ {\it X}}, \textbf{ {\it Y}}, and \textbf{ {\it Z}} in Figure~\ref{fig:BDFrame}. 
This is an inertial frame and in FAST its origin is usually placed at the bottom of the tower as shown.

\subsection{BD Coordinate System}
The BD coordinate system is denoted as $x_1$, $x_2$, and $x_3$ respectively in Figure~\ref{fig:BDFrame}. 
This is an inertial frame used internally in BeamDyn and its origin is placed at the initial position of the blade root point.

\subsection{Blade Reference Coordinate System}
The blade reference coordinate system is denoted as \textbf{ {\it X$_r$}}, \textbf{ {\it Y$_r$}}, and \textbf{ {\it Z$_r$}} in Figure~\ref{fig:BDFrame} at instant $t$. 
This is a floating frame that attaches at the blade root and is rotating with the blade. 
Its origin is at the blade root and the directions of axes following the IEC standard, i.e., \textbf{ {\it Z$_r$}} is pointing along the blade axis from root to tip; \textbf{ {\it Y$_r$}} pointing normally towards the trailing edge of the blade and parallel with the chord line at the zero-twist blade station; and \textbf{ {\it X$_r$}} is orthogonal with the \textbf{ {\it Y$_r$}} and \textbf{ {\it Z$_r$}} axes, such that they form a right-handed coordinate system (pointing nominally downwind). 
We note that the initial blade reference coordinate system, denoted by subscript $r0$, coincides with the BD coordinate system, which is used internally in BeamDyn and introduced in the previous section. 
The axis convention relations between the initial blade reference coordinate system and the BD coordinate system can be found in Table~\ref{tab:IECBD}. 

\subsection{Local blade coordinate system}
The local blade coordinate system is used for some output quantities, for example, the sectional force and moment resultants. This coordinate system is different from the blade reference coordinate system in that its $Z_l$ axis is always tangent to the blade axis. Note that a subscript $l$ denotes the local blade coordinate system.

\section{Geometrically Exact Beam Theory}
The theoretical foundation of BeamDyn is the geometrically exact beam theory. 
This theory features the capability of beams that are initially curved and twisted and subjected to large displacement and rotations. 
Along with a proper two-dimensional (2D) cross-sectional analysis, the coupling effects between all six DOFs, including extension, bending, shear, and torsion, can be captured by GEBT as well . 
The term, ``geometrically exact'' refer to the fact that there is no approximation made on the geometries, including both initial and deformed geometries, in formulating the equations \cite{HodgesBeamBook}.    

 The governing equations of motion for geometrically exact beam theory can be written as \cite{Bauchau:2010}
\begin{align}
	\label{GovernGEBT-1}
	\dot{\underline{h}} - \underline{F}^\prime &= \underline{f} \\
	\label{GovernGEBT-2}
	\dot{\underline{g}} + \dot{\tilde{u}} \underline{h} - \underline{M}^\prime + (\tilde{x}_0^\prime + \tilde{u}^\prime)^T \underline{F} &= \underline{m}
\end{align}
where $\vec{h}$ and $\vec{g}$ are the linear and angular momenta resolved in the inertial coordinate system, respectively; $\vec{F}$ and $\vec{M}$ are the beam's sectional force and moment resultants, respectively; $\vec{u}$ is the one-dimensional (1D) displacement of a point on the reference line; $\vec{x}_0$ is the position vector of a point along the beam's reference line;  and $\vec{f}$ and $\vec{m}$ are the distributed force and moment applied to the beam structure.  
The notation $(\bullet)^\prime$ indicates a derivative with respect to beam axis $x_1$ and $\dot{(\bullet)}$ indicates a derivative with respect to time. 
The tilde operator $(\skew{\bullet})$ defines a skew-symmetric tensor corresponding to the given vector. 
In the literature, it is also termed as ``cross-product matrix''.
For example,
\[
	\skew{n} = 
	     		\begin{bmatrix}
			0 & -n_3 & n_2 \\
			n_3 & 0 & -n_1 \\
			-n_2 & n_1 & 0\\
			\end{bmatrix}	
\]
The constitutive equations relate the velocities to the momenta and the 1D strain measures to the sectional resultants as
\begin{align}
	\label{ConstitutiveMass}
	\begin{Bmatrix}
	\underline{h} \\
	\underline{g}
	\end{Bmatrix}
	= \underline{\underline{\mathcal{M}}} \begin{Bmatrix}
	\dot{\underline{u}} \\
	\underline{\omega}
	\end{Bmatrix} \\
	\label{ConstitutiveStiff}
	\begin{Bmatrix}
	\underline{F} \\
	\underline{M}
	\end{Bmatrix}
	= \underline{\underline{\mathcal{C}}} \begin{Bmatrix}
	\underline{\epsilon} \\
	\underline{\kappa}
	\end{Bmatrix}
\end{align}
where $\underline{\underline{\mathcal{M}}}$ and
$\underline{\underline{\mathcal{C}}}$ are the $6 \times 6$ sectional mass
and stiffness matrices, respectively (note that they are not really tensors);
$\underline{\epsilon}$ and $\underline{\kappa}$ are the 1D strains and
curvatures, respectively; and, $\underline{\omega}$ is the angular velocity
vector that is defined by the rotation tensor $\underline{\underline{R}}$ as
$\underline{\omega} =
axial(\dot{\underline{\underline{R}}}~\underline{\underline{R}}^T)$. 
The axial vector $\vec{a}$ associated with a second-order tensor $\tens{A}$ is denoted $\vec{a}=axial(\tens{A})$ and its components are defined as
\begin{equation}
    \label{axial}
    \vec{a} = axial(\tens{A})=\begin{Bmatrix}
    a_1 \\
    a_2 \\
    a_3
    \end{Bmatrix}
    =\frac{1}{2}
    \begin{Bmatrix}
    A_{32}-A_{23} \\
    A_{13}-A_{31} \\
    A_{21}-A_{12}
    \end{Bmatrix}
\end{equation}
The 1D strain measures are defined as
\begin{equation}
    \label{1DStrain}
    \begin{Bmatrix}
        \vec{\epsilon} \\
        \vec{\kappa}
    \end{Bmatrix}
    =
    \begin{Bmatrix}
        \vec{x}^\prime_0 + \vec{u}^\prime - (\tens{R} ~\tens{R}_0) \bar{\imath}_1 \\
        \vec{k}
    \end{Bmatrix}
\end{equation}
where $\vec{k} = axial [(\tens{R R_0})^\prime (\tens{R R_0})^T]$ is the sectional
curvature vector resolved in the inertial basis; $\tens{R}_0$ is the initial rotation tensor; and $\bar{\imath}_1$ is the unit vector along $x_1$ direction in the inertial basis. 
These three sets of equations, including equations of motion Eq.~\eqref{GovernGEBT-1} and \eqref{GovernGEBT-2}, constitutive equations
Eq.~\eqref{ConstitutiveMass} and \eqref{ConstitutiveStiff}, and kinematical
equations Eq.~\eqref{1DStrain}, provide a full mathematical description of the beam elasticity problems. 

\section{Numerical Implementation with Legendre Spectral Finite Elements}
\label{sec:NumImp}
For a displacement-based finite element implementation, there are six degree-of-freedoms at each node: three displacement components and three rotation components. 
Here we use $\vec{q}$ to denote the elemental displacement array as $\underline{q}=\left[
\underline{u}^T~~\underline{c}^T\right]$ where $\vec{u}$ is the displacement and $\vec{c}$ is the rotation-parameter vector. 
The acceleration array can thus be defined as $\underline{a}=\left[ \ddot{\underline{u}}^T~~ \dot{\underline{\omega}}^T \right]$. 
For nonlinear finite-element analysis, the discretized and incremental forms of displacement, velocity, and acceleration are written as
\begin{align}
	\label{Discretized}
	\underline{q} (x_1) &= \underline{\underline{N}} ~\hat{\underline{q}}~~~~\Delta \underline{q}^T = \left[ \Delta \underline{u}^T~~\Delta \underline{c}^T \right] \\
	\underline{v}(x_1) &= \underline{\underline{N}}~\hat{\underline{v}}~~~~\Delta \underline{v}^T = \left[\Delta \underline{\dot{u}}^T~~\Delta \underline{\omega}^T \right] \\
%	\label{Incremental}
	\underline{a}(x_1) &= \underline{\underline{N}}~ \hat{\underline{a}}~~~~\Delta \underline{a}^T = \left[ \Delta \ddot{\underline{u}}^T~~\Delta \dot{\underline{\omega}}^T \right]	
\end{align}
where $\tens{N}$ is the shape function matrix and $(\hat{\cdot})$ denotes a column matrix of nodal values. 

The displacement fields in an element are approximated as
\begin{align}
    \label{InterpolateDisp}
%    \vec{u}(\xi) &= \sum_{k=1}^{p+1} h^k(\xi) \vec{\hat{u}}^k \\
    \vec{u}(\xi) &=  h^k(\xi) \vec{\hat{u}}^k \\
    \label{InterpolateDispp}
    \vec{u}^\prime(\xi) &=  h^{k\prime}(\xi) \vec{\hat{u}}^k
\end{align}
where $h^k(\xi)$, the component of shape function matrix $\tens{N}$, is the $p^{th}$-order polynomial Lagrangian-interpolant shape function of node $k$, $k=\{1,2,...,p+1\}$, $\vec{\hat{u}}^k$ is the $k^{th}$ nodal value, and $\xi \in \left[-1,1\right]$ is the element natural coordinate.
However, as discussed in \cite{Bauchau-etal:2008}, the 3D rotation field cannot simply be interpolated as the displacement field in the form of
\begin{align}
    \label{InterpolateRot}
    \vec{c}(\xi) &= h^k(\xi) \vec{\hat{c}}^k \\
    \label{InterpolateRotp}
    \vec{c}^\prime(\xi) &= h^{k \prime}(\xi) \vec{\hat{c}}^k 
\end{align}    
where $\vec{c}$ is the rotation field in an element and $\vec{\hat{c}}^k$ is the nodal value at the $k^{th}$ node, for three reasons: 1) rotations do not form a linear space so that they must be  ``composed'' rather than added; 2) a rescaling operation is needed to eliminate the singularity existing in the vectorial rotation parameters; 3) the rotation field lacks objectivity, which, as defined by \cite{Crisfield1999}, refers to the invariance of strain measures computed through interpolation to the addition of a rigid-body motion. 
Therefore, we adopt the more robust interpolation approach proposed by \cite{Crisfield1999} to deal with the finite rotations. Our approach is described as follows
\begin{description}

    \item[Step 1:] Compute the nodal relative rotations, $\vec{\hat{r}}^k$,
by removing the reference rotation, $\vec{\hat{c}}^1$, from the finite
rotation at each node, $\vec{\hat{r}}^k = (\vec{\hat{c}}^{1-}) \oplus
\vec{\hat{c}}^k$. It is noted that the minus sign on $\vec{\hat{c}}^1$ denotes that the relative rotation is calculated by removing the reference rotation from each node.  The composition in that equation is an equivalent of $\tens{R}(\vec{\hat{r}}^k) = \tens{R}^T(\vec{\hat{c}}^1)~\tens{R}(\vec{\vec{c}}^k).$

    \item[Step 2:] Interpolate the relative-rotation field: $\vec{r}(\xi) = h^k(\xi) \vec{\hat{r}}^k$ and $\vec{r}^\prime(\xi) = h^{k \prime}(\xi) \vec{\hat{r}}^k$. Find the curvature field $\vec{\kappa}(\xi) = \tens{R}(\vec{\hat{c}}^1) \tens{H}(\vec{r}) \vec{r}^\prime$, where $\tens{H}$ is the tangent tensor that relates the curvature vector $\vec{k}$ and rotation vector $\vec{c}$ as
\begin{equation}
    \label{Tensor}
    \vec{k} = \tens{H}~ \vec{c}^\prime
\end{equation}

    \item[Step 3:] Restore the rigid-body rotation removed in Step 1: $\vec{c}(\xi) = \vec{\hat{c}}^1 \oplus \vec{r}(\xi)$.
\end{description} 


Note that the relative-rotation field can be computed with respect to any of the nodes of the element; we choose node 1 as the reference node for convenience. 
In the LSFE approach, shape functions (i.e., those composing $\tens{N}$) are $p^{th}$-order Lagrangian interpolants, where nodes are located at the $p+1$ Gauss-Lobatto-Legendre (GLL) points in the $[-1,1]$ element natural-coordinate domain.  
Figure~\ref{fig:N4_lsfe} shows representative LSFE basis functions for  fourth- and eighth-order elements. 
Note that nodes are clustered near element endpoints. More details on the LSFE and its applications can be found in References~\cite{Patera:1984,Ronquist:1987,Sprague:2003,Sprague:2004}.

\begin{figure}[h]
    \centering
    \psfrag{x}[][]{$\xi$}
   \subfigure[$p=4$]{
   \includegraphics[width=0.3\textwidth,clip=true]{\directory N4.pdf}}
   \subfigure[$p=8$]{
   \includegraphics[width=0.3\textwidth,clip=true]{\directory N8.pdf}}
    \caption{Representative $p+1$ Lagrangian-interpolant shape functions in the element natural coordinates for (a) fourth- and (b) eighth-order LSFEs, where nodes are located at the Gauss-Lobatto-Legendre points.}
    \label{fig:N4_lsfe}
\end{figure}

\section{Wiener-Milenkovi\'c Rotation Parameter}
In BeamDyn, the 3D rotations are represented as Wiener-Milenkovi\'c parameters defined in the following equation:
 \begin{equation}
     \vec{c} = 4 \tan\left(\frac{\phi}{4} \right) \bar{n} 
     \label{WMParameter}
 \end{equation}
where $\phi$ is the rotation angle and $\bar{n}$ is the unit vector of the rotation axis. 
It can be observed that the valid range for this parameter is $|\phi| < 2 \pi$. 
The singularities existing at integer multiples of $\pm 2 \pi$ can be removed by a rescaling operation at $\pi$ as:
\begin{equation}
    \label{RescaledWM}
    \vec{r} = \begin{cases}
    4(q_0\vec{p} + p_0 \vec{q} + \tilde{p} \vec{q} ) / (\Delta_1 + \Delta_2), & \text{if } \Delta_2 \geq 0 \\
    -4(q_0\vec{p} + p_0 \vec{q} + \tilde{p} \vec{q} ) / (\Delta_1 - \Delta_2), & \text{if } \Delta_2 < 0
    \end{cases}
\end{equation}
where $\vec{p}$, $\vec{q}$, and $\vec{r}$ are the vectorial parameterization of three finite rotations such that $\tens{R}(\vec{r}) = \tens{R}(\vec{p}) \tens{R}(\vec{q})$, $p_0 = 2 - \vec{p}^T \vec{p}/8$, $q_0 = 2 - \vec{q}^T \vec{q}/8$, $\Delta_1 = (4-p_0)(4-q_0)$, and $\Delta_2 = p_0 q_0 - \vec{p}^T \vec{q}$.
It is noted that the rescaling operation could cause a discontinuity of the interpolated rotation field; therefore a more robust interpolation algorithm
has been introduced in Section~\ref{sec:NumImp} where the rescaling-independent relative-rotation field is interpolated. 

The rotation tensor expressed in terms of Wiener-Milenkovi\'c parameters is
\begin{equation}
   \label{eqn:RotTensorWM}
   \tens{R} (\vec{c}) = \frac{1}{(4-c_0)^2}
   \begin{bmatrix}
    c_0^2 + c_1^2 - c_2^2 - c_3^2 & 2(c_1 c_2 - c_0 c_3) & 2(c_1 c_3 + c_0 c_2) \\
    2(c_1 c_2 + c_0 c_3) & c_0^2 - c_1^2 + c_2^2 - c_3^2 & 2(c_2 c_3 - c_0 c_1) \\
    2(c_1 c_3 - c_0 c_2)  & 2(c_2 c_3 + c_0 c_1) & c_0^2 - c_1^2 - c_2^2 + c_3^2 \\
    \end{bmatrix}
\end{equation}
where $\vec{c} = \left[ c_1~~c_2~~c_3\right]^T$ is the Wiener-Milenkovi\'c parameter and $c_0 = 2 - \frac{1}{8}\vec{c}^T \vec{c}$. 
The relation between rotation tensor and direction cosine matrix (DCM) is
\begin{equation}
    \label{RT2DCM}
    \tens{R} = (\tens{DCM})^T
\end{equation}
Interested users are referred to \cite{Bauchau-etal:2008} and \cite{Wang:GEBT2013} for more details on the rotation parameter and its implementation with GEBT.

\section{Linearization Process}
\label{sec:LinearProcess}
The nonlinear governing equations introduced in the previous section are solved by Newton-Raphson method, where a linearization process is needed. The linearization of each term in the governing equations are presented in this section.

According to \cite{Bauchau:2010}, the linearized governing equations in Eq.~\eqref{GovernGEBT-1} and \eqref{GovernGEBT-2} are in the form of
\begin{equation}
	\label{LinearizedEqn}
	\hat{\underline{\underline{M}}} \Delta \hat{\underline{a}} +\hat{\underline{\underline{G}}} \Delta \hat{\underline{v}}+ \hat{\underline{\underline{K}}} \Delta \hat{\underline{q}} = \hat{\underline{F}}^{ext} - \hat{\underline{F}}
\end{equation} 
where the $\hat{\tens{M}}$, $\hat{\tens{G}}$, and $\hat{\tens{K}}$ are the elemental mass, gyroscopic, and stiffness matrices, respectively;
$\hat{\vec{F}}$ and $\hat{\vec{F}}^{ext}$ are the elemental forces and externally applied loads, respectively. 
They are defined for an element of length $l$ along $x_1$ as follows
\begin{align}
	\label{hatM} 
	\hat{\tens{M}}&= \int_0^l \underline{\underline{N}}^T \mathcal{\underline{\underline{M}}} ~\underline{\underline{N}} dx_1 \\
	\label{hatG}
	\hat{\tens{G}} &= \int_0^l \tens{N}^T \tens{\mathcal{G}}^I~\tens{N} dx_1\\ 
	\label{hatK}
	\hat{\tens{K}}&=\int_0^l \left[ \tens{N}^T (\tens{\mathcal{K}}^I + \mathcal{\tens{Q}})~ \tens{N} + \tens{N}^T \mathcal{\tens{P}}~ \tens{N}^\prime + \tens{N}^{\prime T} \mathcal{\tens{C}}~ \tens{N}^\prime + \tens{N}^{\prime T} \mathcal{\tens{O}}~ \tens{N} \right] d x_1 \\	
	\label{hatF}
	\hat{\vec{F}} &= \int_0^l (\tens{N}^T \vec{\mathcal{F}}^I + \tens{N}^T \mathcal{\vec{F}}^D + \tens{N}^{\prime T} \mathcal{\vec{F}}^C)dx_1 \\
	\label{hatFext}
	\hat{\vec{F}}^{ext}& = \int_0^l \tens{N}^T \mathcal{\vec{F}}^{ext} dx_1 
\end{align}
where $\mathcal{\vec{F}}^{ext}$ is the applied load vector. The new matrix notations in Eqs.~\eqref{hatM} to \eqref{hatFext} are briefly
introduced here. 
$\mathcal{\vec{F}}^C$ and $\mathcal{\vec{F}}^D$ are elastic forces obtained from Eq.~\eqref{GovernGEBT-1} and
\eqref{GovernGEBT-2} as
\begin{align}
	\label{FC}
	\mathcal{\vec{F}}^C &= \begin{Bmatrix}
         \vec{F} \\
	\vec{M}
	\end{Bmatrix} = \tens{\mathcal{C}} \begin{Bmatrix}
	\vec{\epsilon} \\
	\vec{\kappa}
	\end{Bmatrix} \\
	\label{FD}
	\mathcal{\vec{F}}^D & = \begin{bmatrix}
	\underline{\underline{0}} & \underline{\underline{0}}\\
	(\tilde{x}_0^\prime+\tilde{u}^\prime)^T & \underline{\underline{0}}
	\end{bmatrix}
	\mathcal{\vec{F}}^C \equiv \tens{\Upsilon}~ \mathcal{\vec{F}}^C
\end{align}
where $\underline{\underline{0}}$ denotes a $3 \times 3$ null matrix. 
The $\tens{\mathcal{G}}^I$, $\tens{\mathcal{K}}^I$,  $\mathcal{\tens{O}}$, $\mathcal{\tens{P}}$, $\mathcal{\tens{Q}}$, and $\vec{\mathcal{F}}^I$ in Eq.~\eqref{hatG}, Eq.~\eqref{hatK}, and Eq.~\eqref{hatF} are defined as
\begin{align}
        \label{mathcalG}
        \tens{\mathcal{G}}^I &= \begin{bmatrix}
        \tens{0} & (\widetilde{\tilde{\omega} m \vec{\eta}})^T+\tilde{\omega} m \tilde{\eta}^T  \\
        \tens{0} & \tilde{\omega} \tens{\varrho}-\widetilde{\tens{\varrho} \vec{\omega}}
        \end{bmatrix} \\
        \label{mathcalK}
        \tens{\mathcal{K}}^I &= \begin{bmatrix}
        \tens{0} & \dot{\tilde{\omega}}m\tilde{\eta}^T + \tilde{\omega} \tilde{\omega}m\tilde{\eta}^T  \\
        \tens{0} & \ddot{\tilde{u}}m\tilde{\eta} + \tens{\varrho} \dot{\tilde{\omega}}-\widetilde{\tens{\varrho} \vec{\dot{\omega}}}+\tilde{\omega} \tens{\varrho} \tilde{\omega} - \tilde{\omega}  \widetilde{\tens{\varrho} \vec{\omega}}
        \end{bmatrix}\\
	\label{mathcalO}
	\mathcal{\tens{O}} &= \begin{bmatrix}
	\tens{0} & \tens{C}_{11} \tilde{E_1} - \tilde{F} \\
	\tens{0}& \tens{C}_{21} \tilde{E_1} - \tilde{M}
	\end{bmatrix} \\
	\label{mathcalP}
	\mathcal{\tens{P}} &= \begin{bmatrix}
	\tens{0} & \tens{0} \\
	\tilde{F} +  (\tens{C}_{11} \tilde{E_1})^T & (\tens{C}_{21} \tilde{E_1})^T
	\end{bmatrix}  \\
	\label{mathcalQ}
	\mathcal{\tens{Q}} &= \tens{\Upsilon}~ \mathcal{\tens{O}} \\
	\label{mathcalFI}
	\vec{\mathcal{F}}^I &= \begin{Bmatrix}
	m \ddot{\vec{u}} + (\dot{\tilde{\omega}} + \tilde{\omega} \tilde{\omega})m \vec{\eta} \\
	m \tilde{\eta} \ddot{\vec{u}} +\tens{\varrho}\dot{\vec{\omega}}+\tilde{\omega}\tens{\varrho}\vec{\omega}
	\end{Bmatrix}
\end{align}
where $m$ is the mass density per unit length, $\vec{\eta}$ is the location of the sectional center of mass, $\tens{\varrho}$ is the moment of inertia tensor, and the following notations were introduced to simplify the above expressions
\begin{align}
    \label{E1}
    \vec{E}_1 &= \vec{x}_0^\prime + \vec{u}^\prime \\
    \label{PartC}
    \tens{\mathcal{C}} &= \begin{bmatrix}
    \tens{C}_{11} & \tens{C}_{12} \\
    \tens{C}_{21} & \tens{C}_{22}
    \end{bmatrix}
\end{align}

\section{Damping Forces and Linearization}
A viscous damping model has been implemented into BeamDyn to account for the structural damping effect. 
The damping force is defined as 
\begin{equation}
   \label{Damping}
   \vec{f}_d = \tens{\mu}~ \tens{\mathcal{C}} \begin{Bmatrix}
   \dot{\epsilon} \\
   \dot{\kappa}
   \end{Bmatrix}
\end{equation}
where $ \tens{\mu}$ is a user-defined damping-coefficient diagonal matrix. 
The damping force can be recast in two separate parts, like $\vec{\mathcal{F}}^C$ and $\vec{\mathcal{F}}^D$ in the elastic force, as
\begin{align}
   \label{DampingForce-1}
   \vec{\mathcal{F}}^C_d &= \begin{Bmatrix}
   \vec{F}_d \\
   \vec{M}_d
   \end{Bmatrix} \\
   \label{DampingForce-2}
   \vec{\mathcal{F}}^D_d &= \begin{Bmatrix}
    \vec{0} \\
    (\tilde{x}^\prime_0 + \tilde{u}^\prime)^T \underline{F}_d
    \end{Bmatrix}   
\end{align}
The linearization of the structural damping forces are as follows:
\begin{align}
   \label{DampingForceLinear-1}
    \Delta \vec{\mathcal{F}}^C_d &= \tens{\mathcal{S}}_d \begin{Bmatrix}
    \Delta \vec{u}^\prime \\
    \Delta \vec{c}^\prime
    \end{Bmatrix} + \tens{\mathcal{O}}_d \begin{Bmatrix}
    \Delta \vec{u} \\
    \Delta \vec{c}
    \end{Bmatrix} + \tens{\mathcal{G}}_d \begin{Bmatrix}
    \Delta \vec{\dot{u}} \\
    \Delta \vec{\omega}
    \end{Bmatrix}     + \tens{\mu} ~\tens{C} \begin{Bmatrix}
    \Delta \vec{\dot{u}}^\prime \\
    \Delta \vec{\omega}^\prime
    \end{Bmatrix} \\
    \label{DampingForceLinear-2}
    \Delta \vec{\mathcal{F}}^D_d &= \tens{\mathcal{P}}_d \begin{Bmatrix}
    \Delta \vec{u}^\prime \\
    \Delta \vec{c}^\prime
    \end{Bmatrix} + \tens{\mathcal{Q}}_d \begin{Bmatrix}
    \Delta \vec{u} \\
    \Delta \vec{c}
    \end{Bmatrix} + \tens{\mathcal{X}}_d \begin{Bmatrix}
    \Delta \vec{\dot{u}} \\
    \Delta \vec{\omega}
    \end{Bmatrix}     + \tens{\mathcal{Y}}_d \begin{Bmatrix}
    \Delta \vec{\dot{u}}^\prime \\
    \Delta \vec{\omega}^\prime
    \end{Bmatrix}
\end{align}
where the newly introduced matrices are defined as
\begin{align}
    \label{DampingSd}
    \tens{\mathcal{S}}_d &= 
    \tens{\mu} \tens{\mathcal{C}} \begin{bmatrix}
    \tilde{\omega}^T & \tens{0} \\
    \tens{0} & \tilde{\omega}^T
    \end{bmatrix} \\
    \label{DampingOd}
    \tens{\mathcal{O}}_d &= 
    \begin{bmatrix}
    \tens{0} & \tens{\mu} \tens{C}_{11} (\dot{\tilde{u}}^\prime - \tilde{\omega} \tilde{E}_1) - \tilde{F}_d \\
    \tens{0} &\tens{\mu} \tens{C}_{21} (\dot{\tilde{u}}^\prime - \tilde{\omega} \tilde{E}_1) - \tilde{M}_d
    \end{bmatrix} \\
    \label{DampingGd}
    \tens{\mathcal{G}}_d &= 
    \begin{bmatrix}
    \tens{0} & \tens{C}_{11}^T \tens{\mu}^T \tilde{E}_1 \\
    \tens{0} & \tens{C}_{12}^T \tens{\mu}^T \tilde{E}_1 
    \end{bmatrix} \\
    \label{DampingPd}
    \tens{\mathcal{P}}_d &= 
    \begin{bmatrix}
    \tens{0} & \tens{0}  \\
    \tilde{F}_d + \tilde{E}_1^T \tens{\mu} \tens{C}_{11} \tilde{\omega}^T &  \tilde{E}_1^T \tens{\mu} \tens{C}_{12} \tilde{\omega}^T 
    \end{bmatrix} \\
    \label{DampingQd}
    \tens{\mathcal{Q}}_d &= 
    \begin{bmatrix}
    \tens{0} & \tens{0}  \\
    \tens{0} &  \tilde{E}_1^T \tens{O}_{12}
    \end{bmatrix} \\
    \label{DampingXd}
    \tens{\mathcal{X}}_d &= 
    \begin{bmatrix}
    \tens{0} & \tens{0}  \\
     \tens{0} &  \tilde{E}_1^T \tens{G}_{12}
    \end{bmatrix} \\
    \label{DampingYd}
    \tens{\mathcal{Y}}_d &= 
    \begin{bmatrix}
    \tens{0} & \tens{0}  \\
      \tilde{E}_1^T \tens{\mu} \tens{C}_{11} &   \tilde{E}_1^T \tens{\mu} \tens{C}_{12}
    \end{bmatrix} \\
\end{align}
where $\tens{O}_{12}$ and $\tens{G}_{12}$ are the $3 \times 3$ sub matrices of $\mathcal{\tens{O}}$ and $\mathcal{\tens{G}}$ as $\tens{C}_{12}$ in Eq.~\eqref{PartC}.

\section{Convergence Criterion and Generalized-$\alpha$ Time Integrator}
\label{sec:ConvergenceCriterion}
The system of nonlinear equations in Eqs.~\eqref{GovernGEBT-1} and \eqref{GovernGEBT-2} are solved using the Newton-Raphson method with the linearized form in Eq.~\eqref{LinearizedEqn}. 
In the present implementation, an energy-like stopping criterion has been chosen, which is calculated as
\begin{equation}
    \label{StoppingCriterion}
    \| \Delta \mathbf{U}^{(i)T} \left( \fourIdx{t+\Delta t}{}{}{}{\mathbf{R}} -  \fourIdx{t+\Delta t}{}{(i-1)}{}{\mathbf{F}}  \right) \| \leq \| \epsilon_E \left( \Delta \mathbf{U}^{(1)T} \left( \fourIdx{t+\Delta t}{}{}{}{\mathbf{R}} - \fourIdx{t}{}{}{}{\mathbf{F}} \right) \right) \|
\end{equation}
where $\|\cdot\|$ denotes the Euclidean norm, $\Delta \mathbf{U}$ is the incremental displacement vector, $\mathbf{R}$ is the vector of externally applied nodal point loads, $\mathbf{F}$ is the vector of nodal point forces corresponding to the internal element stresses, and $\epsilon_E$ is the user-defined energy tolerance. 
The superscript on the left side of a variable denotes the time-step number (in a dynamic analysis), while the one on the right side denotes the Newton-Raphson iteration number. 
As pointed out by \cite{Bathe-Cimento:1980}, this criterion provides a measure of when both the displacements and the forces are near their equilibrium values. 

Time integration is performed using the generalized-$\alpha$ scheme in BeamDyn, which is an unconditionally stable (for linear systems), second-order accurate algorithm.  
The scheme allows for users to choose integration parameters that introduce high-frequency numerical dissipation. 
More details regarding the generalized-$\alpha$ method can be found in \cite{Chung-Hulbert:1993,Bauchau:2010}.  

\section{Calculation of Reaction Loads}
Since the root motion of the wind turbine blade, including displacements and rotations, translational and angular velocities, and translational and angular accelerates, are prescribed as inputs to BeamDyn either by the driver (in stand-alone mode) or by FAST glue code (in FAST-coupled mode), the reaction loads at the root are needed to satisfy equality of the governing equations. 
The reaction loads at the root are also the loads passing from blade to hub in a full turbine analysis. 

The governing equations in Eq.~\ref{GovernGEBT-1} and \ref{GovernGEBT-2} can be recast in a compact form
\begin{equation}
    \label{CompactGovern}
    \vec{\mathcal{F}}^I - \vec{\mathcal{F}}^{C\prime} + \vec{\mathcal{F}}^D = \vec{\mathcal{F}}^{ext}
\end{equation}
with all the vectors defined in Section~\ref{sec:LinearProcess}.  
At the blade root, the governing equation is revised as
\begin{equation}
    \label{CompactGovernRoot}
    \vec{\mathcal{F}}^I - \vec{\mathcal{F}}^{C\prime} + \vec{\mathcal{F}}^D = \vec{\mathcal{F}}^{ext}+\vec{\mathcal{F}}^R
\end{equation}
where $\vec{\mathcal{F}}^R = \left[ \vec{F}^R~~~\vec{M}^R\right]^T$ is the reaction force vector and it can be solved from Eq.~\ref{CompactGovernRoot} given that the motion fields are known at this point.

\section{Calculation of Blade Loads}
BeamDyn can also calculate the blade loads at each finite element node along the blade axis. 
The governing equation in Eq.~\ref{CompactGovern} are recast as
\begin{equation}
    \label{GovernBF}
     \vec{\mathcal{F}}^A + \vec{\mathcal{F}}^V - \vec{\mathcal{F}}^{C\prime} + \vec{\mathcal{F}}^D = \vec{\mathcal{F}}^{ext}
\end{equation}
where the inertial force vector $\vec{\mathcal{F}}^I$ is split into $\vec{\mathcal{F}}^A$ and $\vec{\mathcal{F}}^V$:
\begin{align}
    \label{mathcalFA}
    \vec{\mathcal{F}}^A &= \begin{Bmatrix}
    m \ddot{\vec{u}} + \dot{\tilde{\omega}}m \vec{\eta}\\
    m \tilde{\eta} \ddot{\vec{u}} + \tens{\rho} \dot{\vec{\omega}}
    \end{Bmatrix} \\
    \label{mathcalFV}
    \vec{\mathcal{F}}^V &= \begin{Bmatrix}
    \tilde{\omega} \tilde{\omega} m \vec{\eta}\\
     \tilde{\omega} \tens{\rho} \vec{\omega}
    \end{Bmatrix} \\    
\end{align}
The blade loads are thus defined as
\begin{equation}
    \label{BladeForce}
    \vec{\mathcal{F}}^{BF} \equiv \vec{\mathcal{F}}^V - \vec{\mathcal{F}}^{C\prime} + \vec{\mathcal{F}}^D
\end{equation}
We note that if structural damping is considered in the analysis, the $\vec{\mathcal{F}}^{C}_d$ and $\vec{\mathcal{F}}^D_d$ are incorporated into the internal elastic forces, $\vec{\mathcal{F}}^C$ and $\vec{\mathcal{F}}^D$, for calculation.
